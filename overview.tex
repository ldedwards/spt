\documentclass[british]{amsart} \usepackage{lmodern}

% Bibliography
\usepackage[backend=biber,style=alphabetic]{biblatex} \addbibresource{spt.bib}

\renewcommand{\sfdefault}{lmss} \renewcommand{\ttdefault}{lmtt}
\usepackage[T1]{fontenc} \usepackage[latin9]{inputenc} \usepackage{verbatim}
\usepackage{amstext} \usepackage{amsthm} \usepackage{amssymb}
\usepackage{subfiles}

\makeatletter
%%%%%%%%%%%%%%%%%%%%%%%%%%%%%% Textclass specific LaTeX commands.
\numberwithin{equation}{section} \numberwithin{figure}{section}
\theoremstyle{plain} \newtheorem{thm}{\protect\theoremname}[section]
\theoremstyle{definition} \newtheorem{defn}[thm]{\protect\definitionname}
\theoremstyle{plain} \newtheorem{assumption}[thm]{\protect\assumptionname}
\theoremstyle{plain} \newtheorem{lem}[thm]{\protect\lemmaname}
\theoremstyle{plain} \newtheorem{prop}[thm]{\protect\propositionname}
\theoremstyle{remark} \newtheorem{rem}[thm]{\protect\remarkname}
\theoremstyle{plain} \newtheorem{cor}[thm]{\protect\corollaryname}

%%%%%%%%%%%%%%%%%%%%%%%%%%%%%% User specified LaTeX commands.
\usepackage{color} \usepackage[colorlinks=true,linkcolor={blue}]{hyperref}

\usepackage{amsfonts}
%\renewenvironment{lyxgreyedout}{\color{red}\bgroup}{\egroup}

\makeatother

\usepackage[british]{babel} \usepackage[babel]{csquotes}
\addto\captionsbritish{\renewcommand{\assumptionname}{Assumption}}
\addto\captionsbritish{\renewcommand{\definitionname}{Definition}}
\addto\captionsbritish{\renewcommand{\lemmaname}{Lemma}}
\addto\captionsbritish{\renewcommand{\propositionname}{Proposition}}
\addto\captionsbritish{\renewcommand{\remarkname}{Remark}}
\addto\captionsbritish{\renewcommand{\theoremname}{Theorem}}
\addto\captionsenglish{\renewcommand{\assumptionname}{Assumption}}
\addto\captionsenglish{\renewcommand{\definitionname}{Definition}}
\addto\captionsenglish{\renewcommand{\lemmaname}{Lemma}}
\addto\captionsenglish{\renewcommand{\propositionname}{Proposition}}
\addto\captionsenglish{\renewcommand{\remarkname}{Remark}}
\addto\captionsenglish{\renewcommand{\theoremname}{Theorem}}
\addto\captionsenglish{\renewcommand{\corollaryname}{Corollary}}
\providecommand{\assumptionname}{Assumption}
\providecommand{\definitionname}{Definition} \providecommand{\lemmaname}{Lemma}
\providecommand{\propositionname}{Proposition}
\providecommand{\remarkname}{Remark} \providecommand{\theoremname}{Theorem}
\providecommand{\corollaryname}{Corollary}

%%%%%%%%%%%%%%%%%%%
\renewcommand{\d}[1]{\mathop{\mathrm{d}{#1}}}
\newcommand{\msquared}{\mathcal{M}^{2}_{0,T}}
\newcommand{\realnumbers}{\mathbb{R}} \newcommand{\ranget}{t\in[0,\infty)}
\newcommand{\filtration}[1]{\mathcal{F}_{#1}}
\newcommand{\defeq}{\mathop{\triangleq}} \newcommand{\almostsurely}{\text{a.s.}}
\newcommand{\abs}[1]{\mathop{|{#1}|}} \newcommand{\market}{\mathcal{M}}
\newcommand{\rangei}{i=1,\dots,n} \newcommand{\measure}{\mathbb{P}}
\newcommand{\probabilityspace}{(\Omega,\filtration,\measure)}
\newcommand{\E}[1]{\mathbb{#1}} \newcommand{\valueprocess}[2]{V^{#1}(#2)}
\newcommand{\norm}[1]{\left\lVert#1\right\rVert}
%%%%%%%%%%%%%%%%%%%

\begin{document}
\title{Functionally Generated Portfolios in Stochastic Portfolio Theory}
\author{Lawrence Edwards} \maketitle

\newpage

\tableofcontents{}

\newpage

%\section{Introduction} \include{introduction} \section{Preliminaries}

%%%%%%%%%%%%%%%%%%%%%%%%%%%%%%%%%%%%%%%%%%%%%%%%%%%%%%%%%%%%%%%%%%%%
\section{Introduction}

Stochastic Portfolio Theory (SPT) is a mathematical theory of how portfolios of
risky assets evolve over time. Unlike Modern Portfolio Theory (MPT) and Capital Asset
Pricing Model (CAPM) theories, SPT is descriptive as opposed to normative and is
therefore consistent with empirical observations of the market.

SPT models assets using continuous semi-martingales to represent the prices of
the assets. By convention the theory uses use the logarithmic representation of
prices and therefore refers to the growth rates of assets and portfolios. This
logarithmic representation is equivalent to the classical arithmetic
representation (i.e. rate of return).

The theory is often presented referring to non-dividend paying stocks for
reasons of simplicity, but the theory has been shown to be compatible with
dividend paying stocks and is in theory applicable to other classes of assets.
Additionally processes with discontinuities, such as jumps have been
incorporated into the theory. The convention is to assume the stocks have a
single share outstanding, so that the prices reflect the total market
capitalisation of the stock.

The theory builds on the concept of \textit{investment strategies}, which are
progressively measurable processes that represent the proportion of the total
wealth invested in each stock at a particular time. \textit{Portfolios} are defined as
investment strategies which are always fully invested in the market. The theory
presents a formula for the (always positive) value process of a strategy,
defined using the sum of the weighted logarithmic returns of a strategy plus a
process, called the \textit{excess growth rate process}, which forms a
fundamental tool in examining how arbitrage opportunities can arise.

An important portfolio, called the \textit{market portfolio} is defined, where the
investment strategy is to hold a fraction of wealth corresponding to a stocks
relative capitalisation. The excess growth rate of this portfolio is interpreted
as the market's "intrinsic volatility" which is shown to lead to arbitrage
opportunities relative to the market when "sufficiency" (boundedness away from
zero) conditions are met.

The theory examines \textit{relative arbitrage} relationships between investment
strategies, both in a \textit{strong arbitrage} and \textit{weak arbitrage}
form. In a link with the Benchmark approach to Mathematical finance
\cite{platen2006}, SPT defines a \textit{numeraire property} of a strategy,
which can be shown to preclude arbitrage over any time horizon.

The cornerstones of SPT are \textit{functionally generated portfolios}, and
the so-called "Fernholz's Master Equation". Functionally generated portfolios are
portfolios generated by a $\mathcal{C}^2$ \textit{generating function} that allow
for the generation of well-performing portfolios that, in general, do not require
estimation of the drifts or volatilities of the stocks. The primary machinery of
SPT in which virtually all relative arbitrages are constructed, the
\textit{Master Equation} measures the performance of any portfolio relative to
the market portfolio. This performance is decomposed into a stochastic part of
infinite variation (written as a function of the market weights), plus a
finite variation component. By choosing a suitable generating function the
stochastic term can be bounded from below, producing relative arbitrage
opportunities.

Some generalisations of functionally generated portfolios have been recently
been proposed, e.g. in \cite{strong2014generalizations} who demonstrates how the
generating function may be dependent on other sources of information. Pal and
Wong in \cite{pal2013} prove that subject to the class of portfolios being only
those generated by current market capitalisations, then a slight generalisation
of functionally generated portfolios are the only class of portfolios that can
lead to relative arbitrage.

Several such classes, corresponding to different assumptions on market
behaviour, have been introduced and studied in SPT; these are:

\begin{enumerate}

\item[\textbf{Diverse}] 

SPT introduces a definition of diversity as being the condition that the market
weight processes weights are bounded from above by a number smaller than one,
effectively meaning that no single company can dominate the entire market
capitalisation.

Diversity is clearly observed in real markets, and its validity is virtually
guaranteed in mature markets by anti-trust regulations. This assumption was
first studied in detail in the context of SPT by Fernholz, Karatzas and Kardaras
\cite{fernholz2005}, who introduced a number of formal definitions of diversity 
and proved that under an additional \textit{non-degeneracy} condition on the 
stock volatilities, relative arbitrages exist in such markets - both over
sufficiently long time horizons and as well as over arbitrarily short time
horizons.

\item[\textbf{Intrinsically Volatile}]

The condition of "sufficiently intrinsic volatility" requires the excess growth
rate of the market portfolio to be bounded away from zero.

Fernholz in \cite{fernholz2002} argued that this condition holds for real
markets, and without any additional assumptions, showed that there exists
relative arbitrage over sufficiently \textit{long} time horizons, where the
duration required to achieve the relative arbitrage depending on the size 
of the lower bound for average market volatility (see
also \cite{fernholz2005relative}.

It remains an open problem whether a relative arbitrage over arbitrarily
\textit{short} time horizons exists. Short horizon arbitrage  has been shown to
exist in volatility-stabilised markets (see \cite{banner2008short} generalised
volatility-stabilised markets (see \cite{pickova2014generalized}), and Markovian
intrinsically volatile models (see Proposition 2 of \cite{fernholz2009}).

\item[\textbf{Rank-based}] 

In Rank-based methods, the drift and volatility processes of each stock
are made to depend on the stocks rank according to its capitalisation.

Fernholz first introduced a framework for studying the performance of portfolios
which put weights on stocks based on their rank instead of their name, allowing
him to theoretically explain certain phenomena observed in real markets in
\cite{fernholz2002}. These models were introduced based on the observation that
the distribution of capital according to rank by capitalisation has been very
stable over the past decades. The dynamics of stocks in these models have
been studied extensively, but the question of existence of (asymptotic) relative
arbitrage has not been addressed yet. A very simple case of a rank-based model,
the Atlas model, was introduced and studied in \cite{banner2005atlas} and
\cite{ichiba2011hybrid}.

\end{enumerate}

In this dissertation, we aim to provide a detailed introduction to the theory
and proofs of SPT, leading to the proof of the Master Equation. Using this as a
foundation we examine the diverse model and show how relative arbitrage
opportunities are created, as well as identifying all of the relevant
assumptions and conditions required for these arbitrages.

Once armed with this theory we then examine various simulations of these
strategies using a proprietary dataset and commercial backtesting framework to
see how these portfolios would perform under real conditions taking into account
trading costs and market impact.

%%%%%%%%%%%%% END INTRODUCTION %%%%%%%%%%%%%%%%%%%%%%%%%

\section{Stochastic Portfolio Theory}

Stochastic Portfolio Theory (SPT) is a mathematical theory of how portfolios of
risky assets evolve over time. Unlike Modern Portfolio Theory (MPT) and Capital Asset
Pricing Model (CAPM) theories, SPT is descriptive as opposed to normative and is
therefore consistent with empirical observations of the market.

%%%%%%%%%%%%%%%%%%%%%%%%%%%%%%%%%%%%%%%%%%%%%%%%%%%%%%%%%%%%%%%%%%%%
\subsection{The Market Model}

We follow the market model introduced by Fernholz (\cite{fernholz1999pgf} and
later in \cite{fernholz2009}) of stock price processes represented by continuous
semimartingales, which is fairly standard in continuous-time financial theory
and investigated in detail in \cite{karatzas1998}.

A number of assumptions are made for clarity of expression, amongst these are:

\begin{itemize}
  \item the number of companies in the market is fixed, and companies do not 
        break up or merge,
  \item the number of shares of a company remains constant,
  \item trading is in continuous time,
  \item dividends are paid continuously,
  \item there are no transaction costs or taxes, and
  \item fractional ownership of shares are allowed.
\end{itemize}

Although most of these assumptions are clearly not based on the empirical facts
of the market, these have been made for clarity and in most cases the theory can
be generalised to include them. Additionally, we also assume that each company
has a single share outstanding, so that the price of the stock is equivalent to
it's market capitalisation.

\begin{defn} [The Equity Market Model]
  \label{def:marketmodel}

  We define market $\market$, with $n$ stocks, and $d$-dimensional independent
  Brownian motion $W(\cdot)$ (with $d \ge n$), defined on a probability space 
  $\probabilityspace$ as

  \begin{gather}
    \label{eq:marketmodel}
    \begin{split}
      \d{B(t)} &= B(t)r(t)\d{t},  
        \quad \ranget, \\
      \d{X_{i}(t)} &= 
            X_{i}(t) 
            \left(
                b_{i}(t)dt + \sum_{\nu=1}^{d} \sigma_{i\nu}(t) dW_{\nu}(t)
            \right),
        \quad \rangei,
        \quad \ranget
    \end{split}
  \end{gather}

  where $W = \left\{ W(t)=(W_{1}(t),...,W_{n}(t)),\filtration{t},\ranget \right\}$
  and $r(\cdot)$ is the interest-rate process for the money-market, $B(0)=1$. The
  price process $X_{i}(t)$ which represents the price of the $i$th stock, where
  $X_{i}(0) = x_{i} > 0$ are the (strictly positive) initial values of the stock
  prices.

  We also assume the $\filtration{}$-progressively measurable $(n \times 1)$
  process $b(\cdot)$ called the \textit{rates of return}, and the $(n \times d)$
  process $\sigma_{i\nu}(t)$ of \textit{volatilities} satisfy the integrability
  conditions: 

  \begin{equation*}
    \int_{0}^{T} 
    \abs{r(t)} 
    \d{t} +
    \sum_{i=1}^{n} \int_{0}^{T} 
      \left( 
          \abs{b_{i}(t)} +
          \sum_{\nu=1}^{d} ( \sigma_{i\nu}(t)^2  ) 
          \right) \d{t} < \infty,
    \quad
    T \in [0, \infty),
    \almostsurely
   \end{equation*}

\end{defn}

The stock price process (\ref{eq:marketmodel}) can be written using the notation 

\begin{equation}
  \label{eq:stockpriceprocessdiff}
    \frac{\d{X_{i}(t)}}{X_{i}(t)} = b_{i}(t)\d{t} + \sum_{\nu=1}^{d} \sigma_{i\nu}(t) dW_{\nu}(t),
  \quad \rangei,
  \quad \ranget
\end{equation}

where this first ratio is defined as

\begin{equation*}
  \frac{\d{X_{i}(t)}}{X_{i}(t)} \defeq 
  \int_{0}^{t} b_{i}(s)\d{s} + 
  \int_{0}^{t} \sum_{\nu=1}^{d} \sigma_{i\nu}(s) dW_{\nu}(s),
  \quad \rangei
\end{equation*}

and is often referred to as the \textit{instantaneous} return on the stock.

This very general setting admits a rich class of continuous-path Ito processes,
with very general distributions: in particular, no Markovian or Gaussian
assumption is imposed. The model has been extended to very general
semimartingale settings; see \cite{kardaras2003} for example.

\subsection{Stocks}

We follow the convention used by Fernholz and use a logarithmic representation
for stocks following directly from Definition \ref{def:marketmodel}.

The logarithmic return (log return) of a financial asset is the change in the
natural logarithm of the asset's value. This is commonly referred to as the
"geometric" or "compound" return. Sometimes, the log return yields a clearer
picture of stock behaviour than is available from the usual arithmetic return
particularly in the case of certain stock portfolios
\cite{fernholz2007statistics}. A logarithmic representation is considered to be
more natural when considering long-term behaviour (see e.g.
\cite{fernholz1982}).

Consider a stock represented by its value $X(t)$ at time $t$ as per Definition
\ref{def:marketmodel}. The value $X$ is assumed to be positive, and varies over
time. If a period of time $dt$ passes and the value of the stock moves from $X$
to $X + \d{X}$, then the arithmetic return over the period is $\frac{\d{X}}{X}$.
The corresponding $\log$ return is defined to be $\d{\log{X}} \defeq \log{(X +
\d{X})} - \log{X}$.

In all cases, the "$log$" is understood to be the natural logarithm.

\begin{prop}
  \cite{fernholz1999pgf}
  \label{thm:logarithmicrepresentation}

  Let $X_{i}(\cdot)$ be stock price processes as defined under market $\market$,
  then

  \begin{equation}
    \label{eq:dlogX}
        \d{\log{X_{i}(t)}} =
          \gamma_{i}(t) \d{t} +
          \sum_{\nu=1}^{d} \sigma_{i\nu}(t) dW_{\nu}(t)
  \end{equation}

  with apostrophe (') denoting vector transposition, and where the process
  $\gamma_{i}(t)$, called the \textit{growth rate}, is defined as

  \begin{equation}
    \label{eq:gamma}
    \gamma_{i}(t)\defeq b_{i}(t)-\frac{1}{2}a_{ii}(t)
  \end{equation}

  We also define the (non-negative definite matrix-valued) \textit{covariance
  process} $a_{ij}(t)$ as

  \begin{gather}
    \label{eq:covarianceprocess}
    \begin{split}
      a_{ij}(t)
        & \defeq \sum_{\nu=1}^{d}\sigma_{i\nu}(t)\sigma_{j\nu}(t) 
%        & = \left( \sigma(t)\sigma'(t) \right)_{ij} \\
%        & = \frac{d}{dt}\left\langle \log X_{i},\log X_{j}\right\rangle(t)
    \end{split}
  \end{gather}

\end{prop}

\begin{proof}

  The proposition is proved by a simple application of It\^{o}'s lemma for
  $\rangei$, using $X = \exp(\log(X))$ and then $u(t) = \log{X}$, we get 
  the following partial derivatives:

  \begin{gather*}
    \begin{split}
      \frac{\partial u}{\partial t} &= 0 \\
      \frac{\partial u}{\partial x} &= \exp{Y(t)}  \\
      \frac{\partial^2 u}{\partial x^2} &= \exp{Y(t)}  \\
    \end{split}
  \end{gather*}

  \begin{equation*}
      \d{X_{i}(t)} = \d{\log{X_{i}(t)}} X_{i}(t) 
          + \frac{1}{2} X_{i}(t) \d{ \langle \log{X_{i}} \rangle_{t} } 
  \end{equation*}

  Rearranging we get

  \begin{gather*}
    \begin{split}
      \d{\log{X_{i}(t)}} &= \frac{\d{X_{i}(t)}}{X_{i}(t)} 
          - \frac{1}{2} \d{ \langle \log{X_{i}} \rangle_{t} } \\
        &= \left( b_{i}(t)\d{t} + \sum_{\nu=1}^{d} \sigma_{i\nu}(t) dW_{\nu}(t) \right)
          - \frac{1}{2} \d{ \langle \log{X_{i}} \rangle_{t} } \\
        &= \left( b_{i}(t)\d{t} + \sum_{\nu=1}^{d} \sigma_{i\nu}(t) dW_{\nu}(t) \right)
          - \frac{1}{2} \left( a_{ii} \right)\d{t} \\
        &= \left( b_{i}(t) - \frac{1}{2}  a_{ii} \right)\d{t}  
          + \sum_{\nu=1}^{d} \sigma_{i\nu}(t) dW_{\nu}(t) \\
        &= \gamma_{i}(t) \d{t}  + \sum_{\nu=1}^{d} \sigma_{i\nu}(t) dW_{\nu}(t ).
    \end{split}
  \end{gather*}

\end{proof}

\begin{thm} [Growth Rate of Stock Price Process]
  \label{thm:growthrate}
  (Equation 1.6 in \cite{fernholz2009})
  The justification for the quantity (\ref{eq:gamma}) being called the 
  \textit{growth rate} is because of the following relationship

  \begin{equation}
    \lim_{T \to \infty} 
      \left( 
      \log{X_{i}(T)} - \int_{0}^{T} \gamma_{i}(t)\d{t} 
      \right) = 0
    \quad \almostsurely
  \end{equation}

  This is valid when the individual stock variances $a_{ii}(\cdot)$ do not
  increase too quickly, e.g. if we have 

  \begin{equation*}
    \lim_{T\to\infty} \left( \frac{\log \log T}{T^2} \int_{0}^{T} a_{ii} \d{t} \right)
    \quad \almostsurely 
  \end{equation*}

  See also Equation 1.6 of \cite{fernholz2009}.

\end{thm}

\begin{proof}
  Build on Corollary 2.2 of \cite{fernholz1999pgf}.
\end{proof}



%%%%%%%%%%%%%%%%%%%%%%%%%%%%%%%%%%%%%%%%%%%%%%%%%%%%%%%%%%%%%%%%%%%%
\subsubsection{Stocks}

SPT models assets using continuous semi-martingales to represent the prices of
the assets. By convention the theory uses use the logarithmic representation of
prices and therefore refers to the growth rates of assets and portfolios. This
logarithmic representation is equivalent to the classical arithmetic
representation (i.e. rate of return).

The theory is often presented referring to non-dividend paying stocks for
reasons of simplicity, but the theory has been shown to be compatible with
dividend paying stocks and is in theory applicable to other classes of assets.
Additionally processes with discontinuities, such as jumps have been
incorporated into the theory. The convention is to assume the stocks have a
single share outstanding, so that the prices reflect the total market
capitalisation of the stock.

%%%%%%%%%%%%%%%%%%%%%%%%%%%%%%%%%%%%%%%%%%%%%%%%%%%%%%%%%%%%%%%%%%%%
\subsubsection{Investment Strategies and Portfolios}

The theory builds on the concept of \textit{investment strategies}, which are
progressively measurable processes that represent the proportion of the total
wealth invested in each stock at a particular time. \textit{Portfolios} are defined as
investment strategies which are always fully invested in the market. The theory
presents a formula for the (always positive) value process of a strategy,
defined using the sum of the weighted logarithmic returns of a strategy plus a
process, called the \textit{excess growth rate process}, which forms a
fundamental tool in examining how arbitrage opportunities can arise.

An important portfolio, called the \textit{market portfolio} is defined, where the
investment strategy is to hold a fraction of wealth corresponding to a stocks
relative capitalisation. The excess growth rate of this portfolio is interpreted
as the market's "intrinsic volatility" which is shown to lead to arbitrage
opportunities relative to the market when "sufficiency" (boundedness away from
zero) conditions are met.

\begin{defn} [Trading Strategy]
  \label{def:tradingstrategy}

  A \textit{trading strategy} is a progressively measurable process $h(\cdot)$ 
  takes values in $\mathbb{R}^{n}$ with a wealth process $V^{w,h}(\cdot)$ 

  \begin{equation*}
    V^{w,h}(t) = \sum_{i=1}^{n} h_{i}(t) X_{i}(t) 
    \quad \ranget
  \end{equation*}

  with $V^{w,h}(0)=w$ for $w > 0$. We also assume a trading strategy $h(\cdot)$  
  satisfies the integrability condition

  \begin{equation*}
    \sum_{i=1}^{n} \int_{0}^{T} 
    \left(
    \abs{(h_{i}(t)b_{i}(t)} + h_{i}^2(t)a_{ii}(t)
      \right) \d{t} < \infty
    \quad \almostsurely.
  \end{equation*}

\end{defn}

\begin{defn} [Self Financing Condition]
  \label{def:selffinancingcondition}  
  A strategy $h(\cdot)$ is called \textit{self-financing} if 
  
  \begin{equation}
    \d{V^{w,h}(t)} = \sum_{i=1}^{n} h_{i}(t) \frac{\d{X_{i}(t)}}{X_{i}(t)}.
  \end{equation}

\end{defn}

\begin{defn} [Portfolio]
  \label{def:portfolio}

  A portfolio is a progressively measurable process $\pi(\cdot)$ uniformly
  bounded in $(t,\omega)$, where $\pi_{i}(t)$ represents the proportion of wealth
  invested in stock $i$ at time $t$, with values in the set $\triangle^{n}$,
  defined as 

  \begin{equation*}
    \triangle^{n} \defeq 
    \left\{
          (\pi_{1}, \dots, \pi_{n}) \in \mathbb{R}^{n} 
          \mid
          \sum_{i=1}^{n} \pi_{i} = 1
    \right\}.
  \end{equation*}

  A negative value for $\pi_{i}(t)$ indicates a short sale, we also define a
  \textit{long only} portfolio $\pi(\cdot)$ as a portfolio where $\pi_{i}(t) \ge
  0$ $\forall \rangei$. We introduce the notation for this set as

  \begin{equation*}
    \triangle_{+}^{n} \defeq 
    \left\{
          (\pi_{1}, \dots, \pi_{n}) \in \triangle^{n} 
          \mid
          \pi_{1} \ge 0, \dots, \pi_{n} \ge 0
          \mid
          \sum_{i=1}^{n} \pi_{i} = 1
    \right\}.
  \end{equation*}

\end{defn}

Every portfolio can therefore be seen to generate a strategy by noting the
relationship between the process $h_{i}(\cdot)$, representing the amount
invested in each stock, and the portfolio weights process $\pi_{i}(\cdot)$ which
represents the proportion of total wealth invested in stock, which is 

\begin{equation}
  \label{eq:wealthinvestedbyportfolio}
  h_i(t) = \pi_{i}(t)V^{w,\pi}(t)
  \quad \rangei.
\end{equation}

\begin{prop} [Wealth Process of a Portfolio]

  The wealth-process $V^{w,\pi}(\centerdot)$ of a portfolio $\pi(\centerdot)$
  with initial wealth $w > 0$ satisfies the stochastic differential equation

  \begin{gather}
    \label{eq:wealthprocess}
    \begin{split}
      \frac{\d{V^{w,\pi}(t)}}{V^{w,\pi}(t)} 
        &= \sum_{i=1}^{n} \pi_{i}(t) \frac{\d{X_{i}(t)}}{X_{i}(t)} \\
        &= b_{\pi}(t)\d{t} + \sum_{\nu=1}^{d} \sigma_{\pi\nu}(t) \d{W_{\nu}(t)}
    \end{split}
  \end{gather}

  where we define the $b(\cdot)$, called the \textit{rate-of-return} of the 
  portfolio $\pi(\cdot)$ as

  \begin{equation*}
    \label{eq:bpi}
    b_{\pi}(t) \defeq \sum_{i=1}^{n} \pi_{i}(t) b_{i}(t)
  \end{equation*}

  and the volatility coefficients $\sigma_{\pi\nu}(\cdot)$ as

  \begin{equation*}
    \label{eq:sigmapi}
    \sigma_{\pi\nu}(\cdot) \defeq \sum_{i=1}^{n} \pi_{i}(t) \sigma_{i\nu}(t)
    \quad \nu=1,\dots,d.
  \end{equation*}

\end{prop}

\begin{proof}

  From the self-financing condition (Definition \ref{def:selffinancingcondition}) 
  and (\ref{eq:wealthinvestedbyportfolio}) we have

  \begin{gather*}
    \begin{split}
      \d{V^{w,\pi}(t)} 
      &= \sum_{i=1}^{n} h_{i}(t) \frac{\d{X_{i}(t)}}{X_{i}(t)} \\
      &= \sum_{i=1}^{n} \pi_{i}(t)V^{w,\pi}(t) \frac{\d{X_{i}(t)}}{X_{i}(t)} \\
      &= V^{w,\pi}(t) \sum_{i=1}^{n} \pi_{i}(t) \frac{\d{X_{i}(t)}}{X_{i}(t)} \\
      \frac{\d{V^{w,\pi}(t)}}{V^{w,\pi}(t)} 
      &= \sum_{i=1}^{n} \pi_{i}(t) \frac{\d{X_{i}(t)}}{X_{i}(t)} \\
    \end{split}
  \end{gather*}

  which completes the proof for the first part of (\ref{eq:wealthprocess}). We
  prove the second equality by using (\ref{eq:stockpriceprocessdiff}), so that

  \begin{gather*}
    \begin{split}
      \frac{\d{V^{w,\pi}(t)}}{V^{w,\pi}(t)}
          & = \sum_{i=1}^{n} \pi_{i}(t) 
          \left(
            b_{i}(t)dt + \sum_{\nu=1}^{d} \sigma_{i\nu}(t) dW_{\nu}(t)
          \right) \\
          & = \sum_{i=1}^{n} \pi_{i}(t) b_{i}(t)dt + 
              \sum_{i=1}^{n} \sum_{\nu=1}^{d} \pi_{i}(t) \sigma_{i\nu}(t) dW_{\nu}(t) \\
          & = \sum_{i=1}^{n} \pi_{i}(t) b_{i}(t)dt + 
              \sum_{\nu=1}^{d} \sum_{i=1}^{n} \pi_{i}(t) \sigma_{i\nu}(t) dW_{\nu}(t) \\
          & = b_{\pi}(t)dt + 
              \sum_{\nu=1}^{d} \sum_{i=1}^{n} \pi_{i}(t) \sigma_{i\nu}(t) dW_{\nu}(t) \\
          & = b_{\pi}(t)dt + 
              \sum_{\nu=1}^{d} \sigma_{\pi\nu}(t) dW_{\nu}(t).
    \end{split}
  \end{gather*}

\end{proof}

Since SPT is interested in the behaviour of portfolios, solutions to
(\ref{eq:wealthprocess}) are of great interest.

\begin{prop} [Strong Solution of Wealth Process]
  \label{prop:solutionofwealthprocess}

%  Originally proved in \cite{fernholz1999pgf}. 

  Let $\pi(\centerdot)$ be a portfolio, then the solution of
  (\ref{eq:wealthprocess}) is

  \begin{equation}
    \label{eq:wealthprocess}
    \d{V^{w,\pi}(t)} =  
        \gamma_{\pi}(t) \d{t} +
        \sum_{\nu=1}^{d} \sigma_{\pi\nu}(t) \d{W_{\nu}(t)}
  \end{equation}

%  or equivalently, as
%
%  \begin{equation}
%    V^{w,\pi}(t) = w \exp{ 
%      \left(
%        \int_{0}^{t} \gamma_{\pi}(u) \d{u} +
%        \sum_{\nu=1}^{d} \int_{0}^{t} \sigma_{\pi\nu}(u) \d{W_{\nu}(u)}
%      \right)},
%  \quad \ranget
%  \end{equation}

  where we define the process $\gamma_{\pi}(\cdot)$, called the \textit{growth
  rate} of the portfolio $\pi(\cdot)$ and the process $\gamma_{\pi}^{*}(\cdot)$,
  called the \textit{excess growth rate} as follows

  \begin{equation*}
    \gamma_{\pi}(t) \defeq 
      \sum_{i=1}^{n} \pi_{i}(t)\gamma_{i}(t) + 
      \gamma_{\pi}^{*}(t)
  \end{equation*}

  \begin{equation*}
    \gamma_{\pi}^{*}(t) \defeq \frac{1}{2} 
        \left(
          \sum_{i=1}^{n} \pi_{i}(t)a_{ii}(t) -
          \sum_{i=1}^{n} \sum_{j=1}^{n} \pi_{i}(t)a_{ij}(t)\pi_{j}(t)
        \right)
  \end{equation*}

\end{prop}

Note that the excess growth rate process $\gamma_{\pi}^{*}(\cdot)$ is the
difference between the weighted sum of the individual stock variances and the
overall portfolio variance, and can therefore be interpreted as the returns due
to diversification. It was proved in \cite{fernholz1999diversity} that for
portfolios with non-negative weights, the excess growth rate is non-negative,
and is positive unless the portfolio consists of a single stock.

\begin{proof}

\end{proof}

\begin{prop} [Growth Rate of Wealth Process]
  \label{thm:wealthgrowthrate}

  The justification for the naming of $\gamma_{\pi}(\cdot)$ as the \textit{growth rate} is because of the following property

  \begin{equation*}
    \lim_{T \to \infty} 
      \left( 
      \log{V^{w,\pi}(T)} - \int_{0}^{T} \gamma_{\pi}(t)\d{t} 
      \right) = 0
    \quad \almostsurely
  \end{equation*}

  which is valid under the condition

  \begin{equation*}
    \lim_{T \to \infty}
      \left(
        \frac{\log \log T}{T^2} \int_{0}^{T} \norm{ a(t) } \d{t}
      \right) = 0,
  % \quad \almostsurely.
  \end{equation*}

\end{prop}

\begin{proof}
  See also Equation 1.14 of \cite{fernholz2009}.
  Build on Corollary 2.2 of \cite{fernholz1999pgf}.

  TODO: Proof.
\end{proof}

To simplify notation slightly, going forward we write $V^{\pi}(\cdot) \defeq V^{1,\pi}$ for initial wealth $w=1$. We also note the following equality analogous to (\ref{eq:wealthprocess})

\begin{prop}

  \begin{equation}
    \label{eq:logvalueprocess}
      \d{\log V^{\pi}(t)} = \gamma_{\pi}^{*}(t)\d{t} + \sum_{i=1}^{n} \pi_{i}(t) \d{\log{X_{i}(t)}}.
  \end{equation}

\end{prop}

\begin{proof}

  TODO: Proof (Using Ito Lemma)

\end{proof}

\newpage




%%%%%%%%%%%%%%%%%%%%%%%%%%%%%%%%%%%%%%%%%%%%%%%%%%%%%%%%%%%%%%%%%%%%
\subsection{Relative Arbitrage}

The theory examines \textit{relative arbitrage} relationships between investment
strategies, both in a \textit{strong arbitrage} and \textit{weak arbitrage}
form.

%%%%%%%%%%%%%%%%%%%%%%%%%%%%%%%%%%%%%%%%%%%%%%%%%%%%%%%%%%%%%%%%%%%%
\subsection{Numeraire Property}

In a link with the Benchmark approach to Mathematical finance
\cite{platen2006}, SPT defines a \textit{numeraire property} of a strategy,
which can be shown to preclude arbitrage over any time horizon.

%%%%%%%%%%%%%%%%%%%%%%%%%%%%%%%%%%%%%%%%%%%%%%%%%%%%%%%%%%%%%%%%%%%%
\section{Functionally Generated Portfolios}

Functionally generated portfolios were introduced in \cite{fernholz1999pgf} as a
generalisation of the diversity-weighted portfolios we will introduce later.

%%%%%%%%%%%%%%%%%%%%%%%%%%%%%%%%%%%%%%%%%%%%%%%%%%%%%%%%%%%%%%%%%%%%
\subsection{Generating Function}

\newcommand{\G}{\mathbf{G}}

\begin{defn} [Generating Function] 
  \label{def:generatingfunction}

  Let $U \subset \triangle^{n}$ be a given open set, then the function
  $\G\in\mathcal{C}^{2}(U,(0,\infty))$ is a generating function for the
  portfolio $\pi(\cdot)$ if $\G$ is such that $x\to
  x_{i}D_{i}\log\G(x)$ is bounded on $U$, and if there exists a
  measurable, adapted process $\mathfrak{g}(\centerdot)$ such that 

  \begin{equation}
    \d{ \log \left( \frac{V^{\pi}(t)}{V^{\mu}(t)} \right) } = 
    \d{ \log \G(\mu(t)) + \mathfrak{g}(t) }
    \quad \ranget
    \quad \almostsurely
  \end{equation}

%  (see 
%    Definition 3.1 in \cite{fernholz1999pgf} and 
%    Definition 2.3.1 in \cite{vervuurt2015})

\end{defn}

\begin{prop} 
  \label{prop:generatingfunction}
  Let a function $\G$ as in Definition \ref{def:generatingfunction}
  generate the portfolio $\pi(\cdot)$, then for each of the portfolio weights
  $\pi_{i}(t)$, $\rangei$

  \begin{equation}
    \pi_{i}(t) = 
      \left( 
        D_{i}\log{\G}(\mu(t)) + 1 - 
          \sum_{j=1}^{n} \mu_{j}(t)D_{j}\log{\G(\mu(t))}
      \right) \mu_{i}(t).
  \end{equation}

\end{prop}

\begin{proof}
  First we verify $\pi(\cdot)$ is a valid portfolio, i.e. fully invested.

   \begin{gather}
    \begin{split}
      \sum_{i=1}^{n} \pi_{i}(t) 
      &= \sum_{i=1}^{n} 
        \left( 
          D_{i}\log{\G}(\mu(t)) + 1 - 
            \sum_{j=1}^{n} \mu_{j}(t)D_{j}\log{\G(\mu(t))}
        \right) \mu_{i}(t) \\
      &= 
        \sum_{i=1}^{n} \mu_{i}(t) D_{i}\log{\G}(\mu(t)) + 
        \sum_{i=1}^{n} \mu_{i}(t)
        \left( 
          1 - \sum_{j=1}^{n} \mu_{j}(t)D_{j}\log{\G(\mu(t))}
        \right) \\
       &= 
        \sum_{i=1}^{n} \mu_{i}(t) D_{i}\log{\G}(\mu(t)) + 
        \sum_{i=1}^{n} \mu_{i}(t) -
        \sum_{i=1}^{n} \mu_{i}(t) \sum_{j=1}^{n} \mu_{j}(t)D_{j}\log{\G(\mu(t))}
\\
       &= 1,
  \end{split}
  \end{gather}

  since by definition $\sum_{i=1}^{n} \mu_{i}(t) = 1$.

  The remainder of the proof is performed by formulating a lemma shown below that
  proves the reverse direction of Proposition \ref{prop:generatingfunction}, (as
  presented in \cite{fernholz2009} and \cite{vervuurt2015}). A proof of the
  reverse direction was performed in \cite{fernholz1999pgf}. 

\end{proof}

%%%%%%%%%%%%%%%%%%%%%%%%%%%%%%%%%%%%%%%%%%%%%%%%%%%%%%%%%%%%%%%%%%%%
\subsection{Fernholz's Master Equation}

\begin{lem} [Fernholz's Master Equation]
  \label{thm:masterequation}

  For a portfolio $\pi(\cdot)$ satisfying \ref{prop:generatingfunction}, then the
  generating function $\G$ generates the portfolio $\pi(\cdot)$, i.e.

  \begin{equation}
    \label{eq:masterequation}
    \log \left( \frac{V^{\pi}(t)}{V^{\mu}(t)} \right) = 
    \log \left( \frac{\mathbf{G}\mu(T)}{\mathbf{G}\mu(0)} \right) + 
      \int_{0}^{T} \mathfrak{g}(t)\d{t}
    \quad \almostsurely
  \end{equation}

  where 

  \begin{equation}
    \mathfrak{g}(t) \triangleq \frac{-1}{\mathbf{G}(\mu(t))}
        \sum_{i=1}^{n} \sum_{j=1}^{n} D_{ij}^{2} \mathbf{G}(\mu(t)) 
        \mu_{i}(t) \mu_{j}(t)
        \tau_{ij}^{\mu}(t
  \end{equation}

  is called the \textit{drift process} of the portfolio $\pi(\cdot)$.

\end{lem}

\begin{proof}

  For the first part of the proof we derive a useful expression for the LHS of
  (\ref{eq:masterequation}, 


\end{proof}


%%%%%%%%%%%%%%%%%%%%%%%%%%%%%%%%%%%%%%%%%%%%%%%%%%%%%%%%%%%%%%%%%%%%
\section{Diverse Models}

%%%%%%%%%%%%%%%%%%%%%%%%%%%%%%%%%%%%%%%%%%%%%%%%%%%%%%%%%%%%%%%%%%%%
\subsection{Relative Arbitrage over Long Horizons}

%%%%%%%%%%%%%%%%%%%%%%%%%%%%%%%%%%%%%%%%%%%%%%%%%%%%%%%%%%%%%%%%%%%%
\subsection{Relative Arbitrage over Short Horizons}

%%%%%%%%%%%%%%%%%%%%%%%%%%%%%%%%%%%%%%%%%%%%%%%%%%%%%%%%%%%%%%%%%%%%
\section{Empirical Analysis}

We present a simulation of DWP  on a realistic simulation platform
with market impact cost model that has been calibrated on billions
of dollars of trading activity on a diverse set of European stocks.
To our knowledge this has not been done before. 

Rebalancing frequency \& turnover constraints (Fernholz gives a formula
to estimate turnover). We also investigate various methods of controlling
turnover (as well as holding constraints which are common in portfolio
management).

Universe is Eurostoxx 600 and we use the historical composition as
at each time point.

Gross Return, Net Return and Sharpe Ratio are given.



\printbibliography

\end{document}
