\documentclass[british]{amsart} \usepackage{lmodern}

% Bibliography
\usepackage[backend=biber,style=alphabetic]{biblatex} \addbibresource{spt.bib}

\renewcommand{\sfdefault}{lmss} \renewcommand{\ttdefault}{lmtt}
\usepackage[T1]{fontenc} \usepackage[latin9]{inputenc} \usepackage{verbatim}
\usepackage{amstext} \usepackage{amsthm} \usepackage{amssymb}
\usepackage{subfiles}

\makeatletter
%%%%%%%%%%%%%%%%%%%%%%%%%%%%%% Textclass specific LaTeX commands.
\numberwithin{equation}{section} \numberwithin{figure}{section}
\theoremstyle{plain} \newtheorem{thm}{\protect\theoremname}[section]
\theoremstyle{definition} \newtheorem{defn}[thm]{\protect\definitionname}
\theoremstyle{plain} \newtheorem{assumption}[thm]{\protect\assumptionname}
\theoremstyle{plain} \newtheorem{lem}[thm]{\protect\lemmaname}
\theoremstyle{plain} \newtheorem{prop}[thm]{\protect\propositionname}
\theoremstyle{remark} \newtheorem{rem}[thm]{\protect\remarkname}
\theoremstyle{plain} \newtheorem{cor}[thm]{\protect\corollaryname}

%%%%%%%%%%%%%%%%%%%%%%%%%%%%%% User specified LaTeX commands.
\usepackage{color} \usepackage[colorlinks=true,linkcolor={blue}]{hyperref}

\usepackage{amsfonts}
%\renewenvironment{lyxgreyedout}{\color{red}\bgroup}{\egroup}

\makeatother

\usepackage[british]{babel} \usepackage[babel]{csquotes}
\addto\captionsbritish{\renewcommand{\assumptionname}{Assumption}}
\addto\captionsbritish{\renewcommand{\definitionname}{Definition}}
\addto\captionsbritish{\renewcommand{\lemmaname}{Lemma}}
\addto\captionsbritish{\renewcommand{\propositionname}{Proposition}}
\addto\captionsbritish{\renewcommand{\remarkname}{Remark}}
\addto\captionsbritish{\renewcommand{\theoremname}{Theorem}}
\addto\captionsenglish{\renewcommand{\assumptionname}{Assumption}}
\addto\captionsenglish{\renewcommand{\definitionname}{Definition}}
\addto\captionsenglish{\renewcommand{\lemmaname}{Lemma}}
\addto\captionsenglish{\renewcommand{\propositionname}{Proposition}}
\addto\captionsenglish{\renewcommand{\remarkname}{Remark}}
\addto\captionsenglish{\renewcommand{\theoremname}{Theorem}}
\addto\captionsenglish{\renewcommand{\corollaryname}{Corollary}}
\providecommand{\assumptionname}{Assumption}
\providecommand{\definitionname}{Definition} \providecommand{\lemmaname}{Lemma}
\providecommand{\propositionname}{Proposition}
\providecommand{\remarkname}{Remark} \providecommand{\theoremname}{Theorem}
\providecommand{\corollaryname}{Corollary}

%%%%%%%%%%%%%%%%%%%
\renewcommand{\d}[1]{\mathop{\mathrm{d}{#1}}}
\newcommand{\msquared}{\mathcal{M}^{2}_{0,T}}
\newcommand{\realnumbers}{\mathbb{R}} \newcommand{\ranget}{t\in[0,\infty)}
\newcommand{\filtration}[1]{\mathcal{F}_{#1}}
\newcommand{\defeq}{\mathop{\triangleq}} \newcommand{\almostsurely}{\text{a.s.}}
\newcommand{\abs}[1]{\mathop{|{#1}|}} \newcommand{\market}{\mathcal{M}}
\newcommand{\rangei}{i=1,\dots,n} \newcommand{\measure}{\mathbb{P}}
\newcommand{\probabilityspace}{(\Omega,\filtration,\measure)}
\newcommand{\E}[1]{\mathbb{#1}} \newcommand{\valueprocess}[2]{V^{#1}(#2)}
%%%%%%%%%%%%%%%%%%%

\begin{document}
\title{Functionally Generated Portfolios in Stochastic Portfolio Theory}
\author{Lawrence Edwards} \maketitle

\newpage

\tableofcontents{}

\newpage

%\section{Introduction} \include{introduction} \section{Preliminaries}

%%%%%%%%%%%%%%%%%%%%%%%%%%%%%%%%%%%%%%%%%%%%%%%%%%%%%%%%%%%%%%%%%%%%
\section{Introduction}

Stochastic Portfolio Theory (SPT) is a mathematical theory of how portfolios of
risky assets evolve over time. Unlike Modern Portfolio Theory (MPT) and Capital Asset
Pricing Model (CAPM) theories, SPT is descriptive as opposed to normative and is
therefore consistent with empirical observations of the market.

SPT models assets using continuous semi-martingales to represent the prices of
the assets. By convention the theory uses use the logarithmic representation of
prices and therefore refers to the growth rates of assets and portfolios. This
logarithmic representation is equivalent to the classical arithmetic
representation (i.e. rate of return).

The theory is often presented referring to non-dividend paying stocks for
reasons of simplicity, but the theory has been shown to be compatible with
dividend paying stocks and is in theory applicable to other classes of assets.
Additionally processes with discontinuities, such as jumps have been
incorporated into the theory. The convention is to assume the stocks have a
single share outstanding, so that the prices reflect the total market
capitalisation of the stock.

The theory builds on the concept of \textit{investment strategies}, which are
progressively measurable processes that represent the proportion of the total
wealth invested in each stock at a particular time. \textit{Portfolios} are defined as
investment strategies which are always fully invested in the market. The theory
presents a formula for the (always positive) value process of a strategy,
defined using the sum of the weighted logarithmic returns of a strategy plus a
process, called the \textit{excess growth rate process}, which forms a
fundamental tool in examining how arbitrage opportunities can arise.

An important portfolio, called the \textit{market portfolio} is defined, where the
investment strategy is to hold a fraction of wealth corresponding to a stocks
relative capitalisation. The excess growth rate of this portfolio is interpreted
as the market's "intrinsic volatility" which is shown to lead to arbitrage
opportunities relative to the market when "sufficiency" (boundedness away from
zero) conditions are met.

The theory examines \textit{relative arbitrage} relationships between investment
strategies, both in a \textit{strong arbitrage} and \textit{weak arbitrage}
form. In a link with the Benchmark approach to Mathematical finance
\cite{platen2006}, SPT defines a \textit{numeraire property} of a strategy,
which can be shown to preclude arbitrage over any time horizon.

The cornerstones of SPT are \textit{functionally generated portfolios}, and
the so-called "Fernholz's Master Equation". Functionally generated portfolios are
portfolios generated by a $\mathcal{C}^2$ \textit{generating function} that allow
for the generation of well-performing portfolios that, in general, do not require
estimation of the drifts or volatilities of the stocks. The primary machinery of
SPT in which virtually all relative arbitrages are constructed, the
\textit{Master Equation} measures the performance of any portfolio relative to
the market portfolio. This performance is decomposed into a stochastic part of
infinite variation (written as a function of the market weights), plus a
finite variation component. By choosing a suitable generating function the
stochastic term can be bounded from below, producing relative arbitrage
opportunities.

Some generalisations of functionally generated portfolios have been recently
been proposed, e.g. in \cite{strong2014generalizations} who demonstrates how the
generating function may be dependent on other sources of information. Pal and
Wong in \cite{pal2013} prove that subject to the class of portfolios being only
those generated by current market capitalisations, then a slight generalisation
of functionally generated portfolios are the only class of portfolios that can
lead to relative arbitrage.

Several such classes, corresponding to different assumptions on market
behaviour, have been introduced and studied in SPT; these are:

\begin{enumerate}

\item[\textbf{Diverse}] 

SPT introduces a definition of diversity as being the condition that the market
weight processes weights are bounded from above by a number smaller than one,
effectively meaning that no single company can dominate the entire market
capitalisation.

Diversity is clearly observed in real markets, and its validity is virtually
guaranteed in mature markets by anti-trust regulations. This assumption was
first studied in detail in the context of SPT by Fernholz, Karatzas and Kardaras
\cite{fernholz2005}, who introduced a number of formal definitions of diversity 
and proved that under an additional \textit{non-degeneracy} condition on the 
stock volatilities, relative arbitrages exist in such markets - both over
sufficiently long time horizons and as well as over arbitrarily short time
horizons.

\item[\textbf{Intrinsically Volatile}]

The condition of "sufficiently intrinsic volatility" requires the excess growth
rate of the market portfolio to be bounded away from zero.

Fernholz in \cite{fernholz2002} argued that this condition holds for real
markets, and without any additional assumptions, showed that there exists
relative arbitrage over sufficiently \textit{long} time horizons, where the
duration required to achieve the relative arbitrage depending on the size 
of the lower bound for average market volatility (see
also \cite{fernholz2005relative}.

It remains an open problem whether a relative arbitrage over arbitrarily
\textit{short} time horizons exists. Short horizon arbitrage  has been shown to
exist in volatility-stabilised markets (see \cite{banner2008short} generalised
volatility-stabilised markets (see \cite{pickova2014generalized}), and Markovian
intrinsically volatile models (see Proposition 2 of \cite{fernholz2009}).

\item[\textbf{Rank-based}] 

In Rank-based methods, the drift and volatility processes of each stock
are made to depend on the stocks rank according to its capitalisation.

Fernholz first introduced a framework for studying the performance of portfolios
which put weights on stocks based on their rank instead of their name, allowing
him to theoretically explain certain phenomena observed in real markets in
\cite{fernholz2002}. These models were introduced based on the observation that
the distribution of capital according to rank by capitalisation has been very
stable over the past decades. The dynamics of stocks in these models have
been studied extensively, but the question of existence of (asymptotic) relative
arbitrage has not been addressed yet. A very simple case of a rank-based model,
the Atlas model, was introduced and studied in \cite{banner2005atlas} and
\cite{ichiba2011hybrid}.

\end{enumerate}

In this dissertation, we aim to provide a detailed introduction to the theory
and proofs of SPT, leading to the proof of the Master Equation. Using this as a
foundation we examine the diverse model and show how relative arbitrage
opportunities are created, as well as identifying all of the relevant
assumptions and conditions required for these arbitrages.

Once armed with this theory we then examine various simulations of these
strategies using a proprietary dataset and commercial backtesting framework to
see how these portfolios would perform under real conditions taking into account
trading costs and market impact.

%%%%%%%%%%%%% END INTRODUCTION %%%%%%%%%%%%%%%%%%%%%%%%%

\section{Stochastic Portfolio Theory}

Stochastic Portfolio Theory (SPT) is a mathematical theory of how portfolios of
risky assets evolve over time. Unlike Modern Portfolio Theory (MPT) and Capital Asset
Pricing Model (CAPM) theories, SPT is descriptive as opposed to normative and is
therefore consistent with empirical observations of the market.

%%%%%%%%%%%%%%%%%%%%%%%%%%%%%%%%%%%%%%%%%%%%%%%%%%%%%%%%%%%%%%%%%%%%
\subsection{The Market Model}

%%%%%%%%%%%%%%%%%%%%%%%%%%%%%%%%%%%%%%%%%%%%%%%%%%%%%%%%%%%%%%%%%%%%
\subsubsection{Stocks}

SPT models assets using continuous semi-martingales to represent the prices of
the assets. By convention the theory uses use the logarithmic representation of
prices and therefore refers to the growth rates of assets and portfolios. This
logarithmic representation is equivalent to the classical arithmetic
representation (i.e. rate of return).

The theory is often presented referring to non-dividend paying stocks for
reasons of simplicity, but the theory has been shown to be compatible with
dividend paying stocks and is in theory applicable to other classes of assets.
Additionally processes with discontinuities, such as jumps have been
incorporated into the theory. The convention is to assume the stocks have a
single share outstanding, so that the prices reflect the total market
capitalisation of the stock.

%%%%%%%%%%%%%%%%%%%%%%%%%%%%%%%%%%%%%%%%%%%%%%%%%%%%%%%%%%%%%%%%%%%%
\subsubsection{Investment Strategies and Portfolios}

The theory builds on the concept of \textit{investment strategies}, which are
progressively measurable processes that represent the proportion of the total
wealth invested in each stock at a particular time. \textit{Portfolios} are defined as
investment strategies which are always fully invested in the market. The theory
presents a formula for the (always positive) value process of a strategy,
defined using the sum of the weighted logarithmic returns of a strategy plus a
process, called the \textit{excess growth rate process}, which forms a
fundamental tool in examining how arbitrage opportunities can arise.

An important portfolio, called the \textit{market portfolio} is defined, where the
investment strategy is to hold a fraction of wealth corresponding to a stocks
relative capitalisation. The excess growth rate of this portfolio is interpreted
as the market's "intrinsic volatility" which is shown to lead to arbitrage
opportunities relative to the market when "sufficiency" (boundedness away from
zero) conditions are met.

%%%%%%%%%%%%%%%%%%%%%%%%%%%%%%%%%%%%%%%%%%%%%%%%%%%%%%%%%%%%%%%%%%%%
\subsection{Relative Arbitrage}

The theory examines \textit{relative arbitrage} relationships between investment
strategies, both in a \textit{strong arbitrage} and \textit{weak arbitrage}
form.

%%%%%%%%%%%%%%%%%%%%%%%%%%%%%%%%%%%%%%%%%%%%%%%%%%%%%%%%%%%%%%%%%%%%
\subsection{Numeraire Property}

In a link with the Benchmark approach to Mathematical finance
\cite{platen2006}, SPT defines a \textit{numeraire property} of a strategy,
which can be shown to preclude arbitrage over any time horizon.

%%%%%%%%%%%%%%%%%%%%%%%%%%%%%%%%%%%%%%%%%%%%%%%%%%%%%%%%%%%%%%%%%%%%
\section{Functionally Generated Portfolios}

Functionally generated portfolios were introduced in \cite{fernholz1999pgf}, and
use a generating function 
is new class of portfolios one can derive a decomposition of their relative
performance into 

and this proves useful in the construction
and study of arbitrages relative to the market. Just as in (7.6), this new
decomposition (11.2) does not involve stochastic integrals, and opens the
possibility for making probability-one comparisons over given, fixed
time-horizons.
This performance is decomposed into a stochastic part of
infinite variation (written as a function of the market weights), plus a
finite variation component. By choosing a suitable generating function the
stochastic term can be bounded from below, producing relative arbitrage
opportunities.


%%%%%%%%%%%%%%%%%%%%%%%%%%%%%%%%%%%%%%%%%%%%%%%%%%%%%%%%%%%%%%%%%%%%
\subsection{Generating Function}

The cornerstones of SPT are \textit{functionally generated portfolios}, and
the so-called "Fernholz's Master Equation". Functionally generated portfolios are
portfolios generated by a $\mathcal{C}^2$ \textit{generating function} that allow
for the generation of well-performing portfolios that, in general, do not require
estimation of the drifts or volatilities of the stocks. 

\begin{defn} [Generating Function] 
  
  \label{def:generatingfunction}

  Let $U \subset \triangle^{n}$ be a given open set, then the function
  $\mathbf{G}\in\mathcal{C}^{2}(U,(0,\infty))$ is a generating function for the
  portfolio $\pi(\cdot)$ if $\mathbf{G}$ is such that $x\to
  x_{i}D_{i}\log\mathbf{G}(x)$ is bounded on $U$, and if there exists a
  measurable, adapted process $\mathfrak{g}(\centerdot)$ such that 

  \begin{equation}
    \d{ \log \left( \frac{V^{\pi}(t)}{V^{\mu}(t)} \right) } = 
    \d{ \log \mathbf{G}(\mu(t)) + \mathfrak{g}(t) }
    \quad \ranget
    \quad \almostsurely
  \end{equation}

  (see 
    Definition 3.1 in \cite{fernholz1999pgf} and 
    Definition 2.3.1 in \cite{vervuurt2015})

\end{defn}

%%%%%%%%%%%%%%%%%%%%%%%%%%%%%%%%%%%%%%%%%%%%%%%%%%%%%%%%%%%%%%%%%%%%
\subsection{Fernholz's Master Equation}

The primary machinery of SPT in which virtually all relative arbitrages are constructed, the
\textit{Master Equation} measures the performance of any portfolio relative to
the market portfolio. This performance is decomposed into a stochastic part of
infinite variation (written as a function of the market weights), plus a
finite variation component. By choosing a suitable generating function the
stochastic term can be bounded from below, producing relative arbitrage
opportunities.

\begin{thm} [Fernholz's Master Equation]
  \label{thm:masterequation}

  \begin{equation}
    \log \left( \frac{V^{\pi}(t)}{V^{\mu}(t)} \right) = 
    \log \left( \frac{\mathbf{G}\mu(T)}{\mathbf{G}\mu(0)} \right) + 
      \int_{0}^{T} \mathfrak{g}(t)\d{t}
    \quad \almostsurely
  \end{equation}

  where 

  \begin{equation}
    \mathfrak{g}(t) \triangleq \frac{-1}{\mathbf{G}(\mu(t))}
        \sum_{i=1}^{n} \sum_{j=1}^{n} D_{ij}^{2} \mathbf{G}(\mu(t)) 
        \mu_{i}(t) \mu_{j}(t)
        \tau_{ij}^{\mu}(t)
  \end{equation}

  is called the \textit{drift process} of the portfolio $\pi(\cdot)$. 

% (See also   Theorem 3.1.5 of \cite{fernholz2002} and Lemma 2.3.2 for a proof).
\end{thm}

\begin{proof}

\end{proof}


%%%%%%%%%%%%%%%%%%%%%%%%%%%%%%%%%%%%%%%%%%%%%%%%%%%%%%%%%%%%%%%%%%%%
\section{Diverse Models}

%%%%%%%%%%%%%%%%%%%%%%%%%%%%%%%%%%%%%%%%%%%%%%%%%%%%%%%%%%%%%%%%%%%%
\subsection{Relative Arbitrage over Long Horizons}

%%%%%%%%%%%%%%%%%%%%%%%%%%%%%%%%%%%%%%%%%%%%%%%%%%%%%%%%%%%%%%%%%%%%
\subsection{Relative Arbitrage over Short Horizons}

%%%%%%%%%%%%%%%%%%%%%%%%%%%%%%%%%%%%%%%%%%%%%%%%%%%%%%%%%%%%%%%%%%%%
\section{Empirical Analysis}

We present a simulation of DWP  on a realistic simulation platform
with market impact cost model that has been calibrated on billions
of dollars of trading activity on a diverse set of European stocks.
To our knowledge this has not been done before. 

Rebalancing frequency \& turnover constraints (Fernholz gives a formula
to estimate turnover). We also investigate various methods of controlling
turnover (as well as holding constraints which are common in portfolio
management).

Universe is Eurostoxx 600 and we use the historical composition as
at each time point.

Gross Return, Net Return and Sharpe Ratio are given.



\printbibliography

\end{document}
