\documentclass[british]{amsart}
\usepackage{lmodern}
\usepackage[backend=biber]{biblatex}
\addbibresource{spt.bib}
\renewcommand{\sfdefault}{lmss}
\renewcommand{\ttdefault}{lmtt}
\usepackage[T1]{fontenc}
\usepackage[latin9]{inputenc}
\usepackage{verbatim}
\usepackage{amstext}
\usepackage{amsthm}
\usepackage{amssymb}

\makeatletter
%%%%%%%%%%%%%%%%%%%%%%%%%%%%%% Textclass specific LaTeX commands.
\numberwithin{equation}{section}
\numberwithin{figure}{section}
\theoremstyle{plain}
\newtheorem{thm}{\protect\theoremname}[section]
\theoremstyle{definition}
\newtheorem{defn}[thm]{\protect\definitionname}
\theoremstyle{plain}
\newtheorem{assumption}[thm]{\protect\assumptionname}
\theoremstyle{plain}
\newtheorem{lem}[thm]{\protect\lemmaname}
\theoremstyle{plain}
\newtheorem{prop}[thm]{\protect\propositionname}
\theoremstyle{remark}
\newtheorem{rem}[thm]{\protect\remarkname}
\theoremstyle{plain}
\newtheorem{cor}[thm]{\protect\corollaryname}

%%%%%%%%%%%%%%%%%%%%%%%%%%%%%% User specified LaTeX commands.
\usepackage{color}
\usepackage[colorlinks=true,linkcolor={blue}]{hyperref}

\usepackage{amsfonts}
%\renewenvironment{lyxgreyedout}{\color{red}\bgroup}{\egroup}

\makeatother

\usepackage[british]{babel}
\usepackage[babel]{csquotes}
\addto\captionsbritish{\renewcommand{\assumptionname}{Assumption}}
\addto\captionsbritish{\renewcommand{\definitionname}{Definition}}
\addto\captionsbritish{\renewcommand{\lemmaname}{Lemma}}
\addto\captionsbritish{\renewcommand{\propositionname}{Proposition}}
\addto\captionsbritish{\renewcommand{\remarkname}{Remark}}
\addto\captionsbritish{\renewcommand{\theoremname}{Theorem}}
\addto\captionsenglish{\renewcommand{\assumptionname}{Assumption}}
\addto\captionsenglish{\renewcommand{\definitionname}{Definition}}
\addto\captionsenglish{\renewcommand{\lemmaname}{Lemma}}
\addto\captionsenglish{\renewcommand{\propositionname}{Proposition}}
\addto\captionsenglish{\renewcommand{\remarkname}{Remark}}
\addto\captionsenglish{\renewcommand{\theoremname}{Theorem}}
\addto\captionsenglish{\renewcommand{\corollaryname}{Corollary}}
\providecommand{\assumptionname}{Assumption}
\providecommand{\definitionname}{Definition}
\providecommand{\lemmaname}{Lemma}
\providecommand{\propositionname}{Proposition}
\providecommand{\remarkname}{Remark}
\providecommand{\theoremname}{Theorem}
\providecommand{\corollaryname}{Corollary}

%%%%%%%%%%%%%%%%%%%
\renewcommand{\d}[1]{\mathop{\mathrm{d}{#1}}}
\newcommand{\realnumbers}{\mathbb{R}}
\newcommand{\ranget}{t\in[0,\infty)}
\newcommand{\filtration}[1]{\mathcal{#1}}
\newcommand{\defeq}{\mathop{\triangleq}}
\newcommand{\almostsurely}{\text{a.s.}}
%%%%%%%%%%%%%%%%%%%

\begin{document}
\title{Examination of practical applications of Functionally Generated 
Portfolios in Stochastic Portfolio Theory}
\author{Lawrence Edwards}
\maketitle

\tableofcontents{}

\section{Introduction}
Stochastic Portfolio Theory (\textbf{SPT}) was first introduced by Fernholz and 
Shay (1982) \cite{fernholz1982}, formalised in Fernholz (1999) 
\cite{fernholz1999} and was further developed by Fernholz (2002) 
\cite{fernholz2002}. It has evolved into a flexible framework for analyzing 
portfolio behavior and equity market structure that has both theoretical and 
practical applications.

From a theoretical standpoint, the framework provides insight into questions 
regarding the behaviour of the market as well as arbitrage, and can be used to 
construct portfolios with controlled behaviour under quite general assumptions. 

The framework does not assume the existence of equivalent local martingale
measures (equivalent to the assumption of \textbf{No Free Lunch with Vanishing 
Risk} (NFLVR)), detailed by Delbaen and Schachermayer \cite{delbaen1994}. In 
fact, most models developed in the framework satisfy only the weaker \textbf{No 
Unbounded Profit with Bounded Risk assumption} (NUPBR), which holds if and only 
if a local martingale deflator exists, which was shown by Kardaras 
\cite{kardaras2012}. 

NUPBR plays the role of Radon-Nikodym derivative in the case that an equivalent
local martingale measure does exist.  Karatzas and Kardaras (2007) proved that 
NUPBR is a necessary and sufficient condition for the log-utility maximization 
problem to have a solution. For an overview of these different notions of 
arbitrage, and their relation to the existence of deflators and equivalent 
measures, we refer the reader to Fontana (2015).

As a practical tool, SPT has been applied to the analysis and optimization of 
portfolio performance and has been the basis of successful equity investment 
strategies for over a decade.

* importance of portfolio construction in finance
* proportion of the market that is passively investing wealth in financial 
market has been growing
* a relatively recent approach to creating portfolios

A thorough survey in Fernholz and Karatzas \cite{fernholz2009} and recently
by Vervuurt \cite{vervuurt2015}.

* relatively general market model is assumed
* continuous-time model with weak no-arbitrage assumptions
* portfolio selection criteria is to outperform a benchmark with probability 1
* relies on a method of \textit{functional generation} to constructs portfolios
that rely deterministically only on the market capitalisation (or equivalently 
prices)

An example is the Diversity weighted portfolios (DWP)
* builds on the concept of diversification leading to superior returns
* we present the in-depth theoretical basis showing that the DWP outperforms the 
market index, including recent enhancements by \cite{vervuurt2016} that
extend the basic formulation to include negative-weighted parameters.

Vervuurt \cite{vervuurt2015} notes in his survey that hardly any of the 
theoretical results in SPT have been tested using real market data. We aim as 
part of this dissertation to provide an in-depth empirical study of the European 
equity market, examining the assumptions and the real world implementation of
such strategies, including frictions such as:
\begin{description}
	\item [Corporate actions] including splits
	\item [Transaction costs] realistic trading costs based on an accurate 
		market impact model
	\item [Market imperfections] including indivisibility of shares, entries 
		and exits from the universe etc.
\end{description}

\section{Preliminaries}
\subsection{Diffusion Processes}

We introduce the stochastic process known as a Brownian motion (also sometimes 
referred to as a Weiner process).

\begin{defn} [Standard Brownian Motion]
A Standard Brownian motion $W$ is a stochastic process which is a mapping 
$W:[0,\infty)\times\Omega\to\mathbb{R}$ for a probability
space $(\Omega,\mathcal{F},P)$ which has the properties:

\begin{enumerate}
	\item $W(0)=0 \quad \almostsurely$,
	\item $W(t)$ has \textbf{independent increments,}
	\item $W(t)$ has \textbf{stationary increments} 
		$W(t)-W(s)\thicksim\mathcal{N}(0,t-s)$for $0\le s<t<\infty,$
	\item the sample trajectories for almost all $\omega\in\Omega$ the paths 
		$t\to W(t,\omega)$ are \textbf{almost surely continuous.}
\end{enumerate}
\end{defn}

\begin{defn} [Variation]
	\label{def:variation}
	For a stochastic process $X(\cdot)$ we write
	\begin{equation*}
		\left< X \right>_{t} \defeq i
			\lim_{\max{(t_{n+1}-t_{n}}\to0}
			\sum \left| \right|
	\end{equation*}
\end{defn}

\begin{thm} [Quadratic variation of Brownian motion]
	\label{thm:quadraticvariationofbrownianmotion}
	\begin{equation}
	\left<W(t)\right>_{t} = t \quad \almostsurely.
	\end{equation}
\end{thm}

\subsection{It\^{o} Calculus}

\newcommand{\msquared}{\mathcal{M}^{2}_{0,T}}

\begin{thm} 
	[Quadratic variation of It\^{o} processes]
	\label{thm:quadraticvariationofitoprocess}
	For an It\^{o} process $X(\cdot)$ 
	\begin{equation*}
		\d{X(t)} = a(t)\d{t} + b(t)\d{W(t)}
	\end{equation*}
	then the quadratic variation of $X$ is
	\begin{equation*}
		\left< X \right>_{t} = \int_{0}^{t} b^2(s)\d{s}.
	\end{equation*}
\end{thm}

\begin{proof}
	We start with the result for continuous, \textit{square-integrable} martingale 
	$I(X) \in \msquared$ given in \cite{shreve2012} (Equation 2.19).
	\begin{equation*}
		\left<I(X)\right>_{t} = \int_{0}^{t} X_{M}^2 d\left<M\right>_{M}.
	\end{equation*}
	and note that since 
\end{proof}

\section{Stochastic Portfolio Theory}

\section{The Market Model}

This section introduces the basic definitions and results that will be used 
throughout the rest of the dissertation.

\subsection{Stocks and Portfolios}

We assume the number of companies in the market is finite and fixed,
the total number of shares of each company is constant, companies
do not merge or break up and trading is continuous in time. In addition
we assume there are no dividends, transaction costs or taxes, as well as there
being no issues with the indivisibility of shares. 

We present the \textit{logarithmic model} from Fernholz (2002) 
\cite{fernholz2002}, Karatzas (2009) \cite{fernholz2009} and Vervuurt (2015)
\cite{vervuurt2015}. We start by defining the stock price process $X(t)$
which represents the price of the stock. To simplify further, we assume that
each company has a single share of stock outstanding. In this case $X(t)$ also
represents the total capitalisation of the company at time $t$.

\begin{defn} [Stock Price Process] 
	\label{def:logpriceprocess}
	A (logarithmic) stock price process is a process that satisfies the stochastic 
	differential equation
	\begin{equation} 
		\label{eq:stockpriceprocess}
		\d{\log{X(t)}} = 
			\gamma(t)\d{t} + 
			\sum_{\nu=1}^{d} \sigma_{\nu}(t){\d{W_{\nu}(t)}}, 
		\quad \ranget
		\quad \almostsurely,
	\end{equation}
\end{defn}

Integrating (\ref{eq:stockpriceprocess}) we get

\begin{equation}
	\log X(t) = \log X(0) + \int_{0}^{t} \gamma(s)\d{s} + 
	\int_{0}^{t} \sum_{\nu=1}^{d} \sigma_{\nu}(s){\d{W_{\nu}(s)}}, \quad 
	\ranget,
\end{equation}

where the positive constant $X(0)$ represents the initial value of the stock. 
This can then easily be written in exponential form as

\begin{equation}
	X(t) = X(0) \exp{ \left(
			\int_{0}^{t} \gamma(s)\d{s} + \int_{0}^{t} 
			\sum_{\nu=1}^{d} \sigma_{\nu}(s){\d{W_{\nu}(s)}} 
			\right)
		}, \quad \ranget.
\end{equation}

The process $\gamma(t)$ is called the \textit{growth rate process} of $X(\cdot)$ 
and the process $\sigma_{\nu}(t)$ are called the \textit{volatility processes} of 
$X(\cdot)$.

\begin{prop} 
	\label{prop:crossvarlogX}
	\cite{fernholz2002} The quadratic variation of 
	the logarithmic stock price process $\log{X(\cdot)}$ satisfies
	\begin{equation}
		\d{\langle \log{X} \rangle_{t}} = \sum_{\nu=1}^{d} \sigma_{\nu}^2\d{t},
		\quad \ranget,
		\quad \almostsurely
	\end{equation}
\end{prop}

\begin{proof}
	From the definition of cross-variation (?? add reference)
	\begin{gather}
		\begin{split} 
			\label{eq:crossvarlogX}
			\langle \log{X} \rangle_{t} 
			 & = \left< \int_{0}^{t} \sum_{\nu=1}^{d} \sigma_{\nu}(s) \d{W_{\nu}(s)} \right>_{t} \\
			 & = \left< \sum_{\nu=1}^{d} \int_{0}^{t} \sigma_{\nu}(s) \d{W_{\nu}(s)} \right>_{t} \\
			 & = \sum_{\nu=1}^{d} \left< \int_{0}^{t} \sigma_{\nu}(s) \d{W_{\nu}(s)} \right>_{t} \\
			 & = \sum_{\nu=1}^{d} \int_{0}^{t} \sigma_{\nu}^{2}(s) \d{\langle W_{\nu}(s) \rangle } \\
			 & = \int_{0}^{t} \sum_{\nu=1}^{d} \sigma_{\nu}^{2}(s) \d{s}.
		\end{split}
	\end{gather}
	Then from (\ref{eq:crossvarlogX}) we have
	\begin{gather}
		\begin{split}
			\d{\langle \log{X} \rangle_{t}}
				& = \d{\left<\int_{0}^{t} \sum_{\nu=1}^{d} \sigma_{\nu}(s) \d{W_{\nu}(s)}\right>_{t}}\\
				& = \d{\left(\int_{0}^{t} \sum_{\nu=1}^{d} \sigma_{\nu}^{2}(s) \d{s}\right)} \\
				& = \sum_{\nu=1}^{d} \sigma_{\nu}^{2}(t) \d{t}.
		\end{split}
	\end{gather}
\end{proof}

We examine the relationship between the standard model of a price process 
$X(\cdot)$ and the logarithmic processes defined above 
(\ref{def:logpriceprocess}). 

\begin{thm} \cite{fernholz2002} The stock price process $X(\cdot)$ satisfies
	\begin{gather}
		\begin{split}
			\d{X(t)}
				& = \alpha(t)X(t)\d{t} + X(t)\sum_{\nu=1}^{d} \sigma_{\nu}(t){\d{W_{\nu}(t)}}, \quad \ranget, \\
				& = X(t) \left( \alpha(t)\d{t} + \sum_{\nu=1}^{d} 
							\sigma_{\nu}(t){\d{W_{\nu}(t)}} \right), \quad \ranget,
		\end{split}
	\end{gather}
	where $W(\cdot)=(W_{1}(\cdot),\dots,W_{d}(\cdot))$ is a $d$-dimensional 
	standard Brownian motion and here we define $\alpha(t)$, a 	measurable and 
	adapted process called the \textit{rate of return process} that satisfies 
	\begin{equation}
		\label{eq:alpha}
		\alpha(t) = \gamma(t) + \frac{1}{2} \sum_{\nu=1}^{d} \sigma_{\nu}^2(t),
		\quad \ranget.
	\end{equation}
\end{thm}

\begin{proof}
	The proof involves an application of It\^{o} formula (?? definition), where we 
	define $Y(t) \defeq \log{X(t)}$ and apply the formula.
	We note the expressions:
	\begin{align*}
		\frac{\partial F}{\partial t}(t,y) &= 0, \\
		\frac{\partial F}{\partial y}(t,y) &= \exp(Y(t)), \\
		\frac{\partial^2 F}{\partial y^2}(t,y) &= \exp(Y(t)).
	\end{align*}
	Using the above expressions in the formula yields
	\begin{equation*}
		\d{F(t,Y(t))} = \exp(Y(t))\d{Y(t)} 
				+ \frac{1}{2}\exp(Y(t))\d{\langle Y \rangle_{t}},
	\end{equation*}
	Therefore since $Y(t) \defeq \log{X(t)}$ 
	\begin{align*}
		\d{X(t)} 
		& = 
						\exp(\log{X(t)})\d{\log{X(t)}} + 
						\frac{1}{2}\exp(\log{X(t)})\d{\langle \log{X} \rangle_{t}} \\
		& = X(t)\d{\log{X(t)}} + \frac{1}{2}X(t)\d{\langle \log{X} \rangle_{t}}.
	\end{align*}
	Recalling the definition of the \textit{log price process} $\log{X(t)}$ 
	(\ref{def:logpriceprocess}) and the \textit{rate of return} process 
	$\alpha(t)$ (\ref{eq:alpha}), and using the result of the proposition 
	(\ref{prop:crossvarlogX}), we get
	\begin{align*}
		\d{X(t)} 
					& = X(t)
					\left( 
						\gamma(t)\d{t} + \sum_{\nu=1}^{d} \sigma_{\nu}(t){\d{W_{\nu}(t)}}, 
					\right) + 
					\frac{1}{2} X(t) \sum_{\nu=1}^{d} \sigma_{\nu}^2\d{t} \\
					& =
					X(t)
						\left(
						\left( 
							\gamma(t) + \frac{1}{2} X(t) \sum_{\nu=1}^{d} \sigma_{\nu}^2
						\right)\d{t} + 
					  \sum_{\nu=1}^{d} \sigma_{\nu}(t){\d{W_{\nu}(t)}}
						\right) \\
					& =	X(t) \left(\alpha(t)\d{t} + \sum_{\nu=1}^{d} \sigma_{\nu}(t) \d{W_{\nu}(t)} \right)
	\end{align*}

\end{proof}

\begin{cor} 
	[Instantaneous Rate of Return]
	The instantaneous rate of return (or, arithmetic return) for the 
	classical equity market model is given by
\begin{equation}
	\frac{\d{X(t)}}{X(t)} = \alpha(t)\d{t} + \sum_{\nu=1}^{d}\sigma_{\nu}(t){\d{W_{\nu}(t)}},
	\quad \ranget.
\end{equation}
\end{cor}

\begin{proof}

\end{proof}

\begin{rem} [Existence of Riskless Asset]
	Note that we implicitly assume the existence of a riskless asset 
	$X_{0}(t)\defeq 1,\ranget$, and without loss of generality we assume a zero 
	interest rate by discounting the stock prices by the bond price.
\end{rem}

\subsection{Portfolios and Wealth}

\subsubsection{The Market Portfolio}
One of the most researched applications of Stochastic Portfolio Theory is that 
of portfolio construction. Here the objective of portfolio construction is to 
selection a portfolios is to exploit arbitrages that may be present in the 
market. Typically this performance is measured with respect to a benchmark, as 
in the Benchmark Approach of Platen and Heath \cite{platen2006}.

\begin{defn}[Market Portfolio]
We introduce the Market Portfolio, as the process of market weights $\mu(\cdot)$ 
defined by \begin{equation}
	\mu_{i}(t)\triangleq\frac{X_{i}(t)}{X(t)},\;\;\;\;X(t)\triangleq\sum_{i=1}^{n}X_{i}(t).
\end{equation}
\end{defn}

\subsection{Relative Arbitrage}
\subsection{Functionally Generated Portfolios}
\subsection{Rank-based Portfolios}
\subsection{Trading Strategies}

\section{Diversity Weighted Portfolios}

Diversity is clearly observed in real markets, and its validity is guaranteed by 
the fact that anti-trust regulations are typically in place. This assumption was 
first studied in detail in the context of SPT by Fernholz, Karatzas and Kardaras 
\cite{fernholz2005}, who defined and studied different forms of diversity, and 
proved that under an additional non-degeneracy condition on the stock 
volatilities, relative arbitrages exist in such markets - both over sufficiently 
long time horizons, as well as over arbitrarily short time horizons.

\printbibliography

\end{document}
