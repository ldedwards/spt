\documentclass[british]{amsart} \usepackage{lmodern}

% Bibliography
\usepackage[backend=biber,style=alphabetic]{biblatex} 
\addbibresource{spt.bib}
%\usepackage[style=alphabetic]{biblatex} 
%\addbibresource{spt.bib}

\renewcommand{\sfdefault}{lmss} \renewcommand{\ttdefault}{lmtt}
\usepackage[T1]{fontenc} \usepackage[latin9]{inputenc} \usepackage{verbatim}
\usepackage{amstext} \usepackage{amsthm} \usepackage{amssymb}
\usepackage{subfiles}
\usepackage{graphicx}
\usepackage{booktabs}
\usepackage{pdflscape}
\usepackage{afterpage}

\makeatletter
%%%%%%%%%%%%%%%%%%%%%%%%%%%%%% Textclass specific LaTeX commands.
\numberwithin{equation}{section} \numberwithin{figure}{section}
\theoremstyle{plain} \newtheorem{thm}{\protect\theoremname}[section]
\theoremstyle{definition} \newtheorem{defn}[thm]{\protect\definitionname}
\theoremstyle{plain} \newtheorem{assumption}[thm]{\protect\assumptionname}
\theoremstyle{plain} \newtheorem{lem}[thm]{\protect\lemmaname}
\theoremstyle{plain} \newtheorem{prop}[thm]{\protect\propositionname}
\theoremstyle{remark} \newtheorem{rem}[thm]{\protect\remarkname}
\theoremstyle{plain} \newtheorem{cor}[thm]{\protect\corollaryname}

%%%%%%%%%%%%%%%%%%%%%%%%%%%%%% User specified LaTeX commands.
\usepackage{color} \usepackage[colorlinks=true,linkcolor={blue}]{hyperref}

\usepackage{amsfonts}
%\renewenvironment{lyxgreyedout}{\color{red}\bgroup}{\egroup}

\makeatother

\usepackage[british]{babel} \usepackage[babel]{csquotes}
\addto\captionsbritish{\renewcommand{\assumptionname}{Assumption}}
\addto\captionsbritish{\renewcommand{\definitionname}{Definition}}
\addto\captionsbritish{\renewcommand{\lemmaname}{Lemma}}
\addto\captionsbritish{\renewcommand{\propositionname}{Proposition}}
\addto\captionsbritish{\renewcommand{\remarkname}{Remark}}
\addto\captionsbritish{\renewcommand{\theoremname}{Theorem}}
\addto\captionsenglish{\renewcommand{\assumptionname}{Assumption}}
\addto\captionsenglish{\renewcommand{\definitionname}{Definition}}
\addto\captionsenglish{\renewcommand{\lemmaname}{Lemma}}
\addto\captionsenglish{\renewcommand{\propositionname}{Proposition}}
\addto\captionsenglish{\renewcommand{\remarkname}{Remark}}
\addto\captionsenglish{\renewcommand{\theoremname}{Theorem}}
\addto\captionsenglish{\renewcommand{\corollaryname}{Corollary}}
\providecommand{\assumptionname}{Assumption}
\providecommand{\definitionname}{Definition} \providecommand{\lemmaname}{Lemma}
\providecommand{\propositionname}{Proposition}
\providecommand{\remarkname}{Remark} \providecommand{\theoremname}{Theorem}
\providecommand{\corollaryname}{Corollary}

%%%%%%%%%%%%%%%%%%%
\renewcommand{\d}[1]{\mathop{\mathrm{d}{#1}}}
\newcommand{\msquared}{\mathcal{M}^{2}_{0,T}}
\newcommand{\realnumbers}{\mathbb{R}} \newcommand{\ranget}{t\in[0,\infty)}
\newcommand{\filtration}[1]{\mathcal{F}_{#1}}
\newcommand{\defeq}{\mathop{\triangleq}} \newcommand{\almostsurely}{\text{a.s.}}
\newcommand{\abs}[1]{\mathop{|{#1}|}} \newcommand{\market}{\mathcal{M}}
\newcommand{\rangei}{i=1,\dots,n} \newcommand{\measure}{\mathbb{P}}
\newcommand{\probabilityspace}{(\Omega,\filtration,\measure)}
\newcommand{\E}[1]{\mathbb{#1}} \newcommand{\valueprocess}[2]{V^{#1}(#2)}
\newcommand{\norm}[1]{\left\lVert#1\right\rVert}
\newcommand{\V}{V^{w,\pi}}
\newcommand{\Vmu}{V^{\mu}}
%%%%%%%%%%%%%%%%%%%

\begin{document}
\title{Empirical analysis of the performance of Diverse portfolios}
\author{Lawrence Edwards} \maketitle

\newpage

\tableofcontents{}

\listoftables

\newpage

%%%%%%%%%%%%%%%%%%%%%%%%%%%%%%%%%%%%%%%%%%%%%%%%%%%%%%%%%%%%%%%%%%%%
\section{Introduction}

Stochastic Portfolio Theory (SPT) is a mathematical theory of how portfolios of
risky assets evolve over time. Unlike Modern Portfolio Theory (MPT) and Capital
Asset Pricing Model (CAPM) theories, SPT is descriptive as opposed to normative
and is therefore consistent with empirical observations of the market.

SPT models prices of assets using continuous semi-martingales. By convention the
theory uses the logarithmic representation of prices and therefore refers to the
growth rates of assets and portfolios. In this context, the logarithmic
representation allows us to apply the tools of stochastic calculus in a more
convenient manner. 

The theory is often presented referring to non-dividend paying stocks for
reasons of simplicity, but the theory has been shown to be compatible with
dividend paying stocks and is applicable to other classes of assets.
Additionally processes with discontinuities, such as jumps have been
incorporated into the theory. The convention is to assume the stocks have a
single share outstanding, so that the prices reflect the total market
capitalisation of the stock.

The theory builds on the concept of \textit{investment strategies}, which are
defined as progressively measurable processes that represent the proportion of
the total wealth invested in each stock at a particular time.
\textit{Portfolios} are defined similarly to investment strategies, but 
the weights represent the fraction of wealth invested in a particular stock at
any point in time, and are always fully invested in the market. The theory
presents a formula for the (always positive) value process of a strategy,
defined using the sum of the weighted logarithmic returns of a strategy plus a
process, called the \textit{excess growth rate process}, which forms a
fundamental tool in examining how arbitrage opportunities can arise.

An important portfolio, called the \textit{market portfolio} is defined, where
the investment strategy is to hold a fraction of wealth corresponding to a
stocks relative capitalisation. The excess growth rate of this portfolio is
interpreted as the market's intrinsic volatility which is shown to lead to
arbitrage opportunities relative to the market when sufficiency (boundedness
away from zero) conditions are met.

The theory examines \textit{relative arbitrage} relationships between investment
strategies, both in a \textit{strong arbitrage} and \textit{weak arbitrage}
form. In a link with the Benchmark approach to Mathematical finance
\cite{platen2006}, SPT defines a \textit{numeraire property} of a strategy,
which can be shown to preclude arbitrage over any time horizon.

The cornerstones of SPT are \textit{functionally generated portfolios}, and
Fernholz's Master Equation. Functionally generated portfolios are
portfolios generated by $\mathcal{C}^2$ \textit{generating functions} that allow
for the formulaic construction of well performing portfolios that
do not require estimation of the drifts or volatilities of the stocks. The
primary machinery of SPT in which all relative arbitrages are
constructed, called the \textit{Master Equation} measures the performance of any
portfolio relative to the market portfolio. This performance is decomposed into
a stochastic part of infinite variation (written as a function of the market
weights), plus a finite variation component. By choosing a suitable generating
function the stochastic term can be bounded from below, producing relative
arbitrage opportunities.

Some generalisations of functionally generated portfolios have recently been
proposed, e.g. Strong et. al. \cite{strong2014generalizations} demonstrate
how the generating function may be dependent on other sources of information,
such as sentiment or other market factors. Pal and Wong \cite{pal2013} prove
that subject to the class of portfolios being only those generated by current
market capitalisations, then a slight generalisation of functionally generated
portfolios are the only class of portfolios that permit relative arbitrage.

Several such classes, corresponding to different assumptions on market
behaviour, have been introduced and studied in SPT; these are:

\begin{itemize}

\item \textbf{Diverse:} SPT introduces a definition of diversity as being the
condition that the market weight processes weights are bounded from above by a
number smaller than one, effectively meaning that no single company can dominate
the entire market capitalisation.

Diversity is clearly observed in real markets, and its validity is virtually
guaranteed in mature markets by antitrust regulations. This assumption was
first studied in detail using the tools of SPT by Fernholz, Karatzas and
Kardaras \cite{fernholz2005}, who introduced formal definitions of diversity and
proved that under an additional \textit{non-degeneracy} condition on the stock
volatilities, relative arbitrages exist in such markets  both over sufficiently
long time horizons and as well as over arbitrarily short time horizons.

\item \textbf{Intrinsically Volatile:} The condition of "sufficiently intrinsic
volatility" requires the excess growth rate of the market portfolio to be
bounded away from zero.

Fernholz in \cite{fernholz2002} argued that this condition holds for real
markets, and without any additional assumptions, showed that there exists
relative arbitrage over sufficiently \textit{long} time horizons, where the
duration required to achieve the relative arbitrage depends on the magnitude of
the lower bound for average market volatility (see also
\cite{fernholz2005relative}.

It remains an open problem whether a relative arbitrage over arbitrarily
\textit{short} time horizons exists. Short horizon arbitrage  has been shown to
exist in volatility stabilised markets (\cite{banner2008short} generalised
volatility stabilised markets (\cite{pickova2014generalized}), and Markovian
intrinsically volatile models (Proposition 2 of \cite{fernholz2009}).

\item \textbf{Rank-based:} In Rank-based methods, the drift and volatility
processes of each stock are made to depend on the stocks rank according to its
capitalisation.

Fernholz first introduced a framework for studying the performance of portfolios
which put weights on stocks based on their rank instead of their name, allowing
him to explain certain phenomena observed in real markets in
\cite{fernholz2002}. These models were introduced based on the observation that
the distribution of capital according to rank by capitalisation has been very
stable over the past decades. The dynamics of stocks in these models have been
studied extensively, but the question of existence of (asymptotic) relative
arbitrage has not been addressed yet. A very simple case of a rank-based model,
the Atlas model, was introduced and studied in \cite{banner2005atlas} and
\cite{ichiba2011hybrid}.

\end{itemize}

In this dissertation, we aim to provide a detailed introduction to the theory
and proofs of SPT, leading to the proof of the Master Equation. Using this as a
foundation we examine the diverse model and show how relative arbitrage
opportunities are created, as well as identifying all of the relevant
assumptions and conditions required for these arbitrages.

Once armed with this theory we then examine various simulations of these
strategies using a proprietary dataset and commercial backtesting framework to
see how these portfolios would perform under real conditions taking into account
trading costs and market impact.

% TODO: Expand introduction with details on simulation
% We carry out an empirical study of the constructed portfolios in Section
%6, demonstrating that our adjustments of the diversity-weighted portfolio would
%have outperformed quite considerably the S&P 500 index, if imple mented on the
%index constituents over the 25 year period between January 1, 1990 and December
%31, 2014. In this study we incorporate 0.5% proportional transaction costs,
%delistings due to bankruptcies, mergers and acquisitions, and distributions such
%as dividends. We compute the relative returns of our portfolios over this
%period, as well as the Sharpe ratios relative to the market index. We discuss
%the results and possible future work in Section 7.

%%%%%%%%%%%%% END INTRODUCTION %%%%%%%%%%%%%%%%%%%%%%%%%
\newpage
\section{Preliminaries}

This chapter reviews some results and definitions from and fixes notation for
the remainder of the paper.

\subsection{Log-Returns}

We define the daily log-return of a stock price as the daily increment
of the natural logarithm of this price. If $X(t)$ represents the stock
price at time $t$, then the log-return $R(t)$ is defined as

\begin{equation}
  R(t) 
    \defeq \log{X(t+\Delta t)} - \log{X(t)} 
    = \log \left( \frac{\log{X(t+\Delta t)}}{\log{X(t)}} \right),
  \quad \ranget.
\end{equation}

\subsection{Probability and Stochastic Processes}

This section provides a concise overview of stochastic calculus for
diffusion processes. The model for continuous stock prices presented
here has an extensive literature can be found in \cite{shreve1991}.

Throughout, we work on a filtered probability space denoted
by $(\Omega,\mathcal{F},\left\{ \mathcal{F}_{t}\right\} _{t\ge0},P)$
where:

\begin{itemize}
  \item $\Omega$ denotes the space of all events,
  \item $\mathcal{F}$ denotes the set of all measurable events,
  \item $\{ \mathcal{F}_{t}\}_{t\ge0}$ denotes the filtration
         a family of $\sigma$-algebras contained in $\mathcal{F}$ with
        the property that if $s<t$ then $\mathcal{F}_{s}\subset\mathcal{F}_{t}$. 
\end{itemize}

All the information accumulated until time $t$ is contained by the
$\sigma$-field $\mathcal{F}_{t}$. This means that $\mathcal{F}_{t}$ contains 
all of the information of which events have already occurred until time $t$, 
and which did not. 

\begin{defn} [Random Variable]
 A random variable $X$ is a function that assigns a numeric value to each state of 
 the world, $X(t):\Omega\to\mathbb{R}$, such that the values taken by
 $X$ are known to someone who has access to the information in $\mathcal{F}$. 
  More precisely,

  \begin{equation}
    \left\{
      \omega \in \Omega : a < X(\omega) < b
    \right\} \in \mathcal{F}
  \end{equation}

  In other words we say that $X$ is an $\mathcal{F}$-measurable function. 
  If a random variable $X$ is \textit{measurable} we say it is \textit{predictable}.

\end{defn}

\begin{defn} [Stochastic Processes]
  The modelling of random assets is based on stochastic processes, which are
  families $X(t)$  of random variables $X(t):\Omega\to\mathbb{R}$ for $t\in[0,T]$,
  where $X(0)$ is constant.
\end{defn}

\begin{defn} [Filtration Generated by a Process]
  The filtration generated by a stochastic process $X(t)$
  is a family of $\sigma$-fields, defined as, 
  
  \begin{equation}
    \mathcal{F}^{X}(t) \defeq \sigma \left\{ X(s) \mid 0 \le s \le t \right\} 
    \quad \ranget.
  \end{equation}

\end{defn}

\begin{defn} [Adapted]
  A process $\left\{ X(t),\mathcal{F}_{t},t\in[0,T]\right\} $ is adapted
  to a filtration $\mathcal{F}$ if $X(t)$ is $\mathcal{F}_{t}$-measurable
  for all $t\in[0,T]$. 
\end{defn}

\begin{defn} [Brownian Motion]
  A Standard Brownian motion $W$ is a stochastic process which
  is a mapping $W:[0,\infty)\times\Omega\to\mathbb{R}$ for a probability
  space $(\Omega,\mathcal{F},P)$ which has the properties:

  \begin{enumerate}
    \item $W(0)=0$, a.s.,
    \item $W(t)$ has independent increments,
    \item $W(t)$ has stationary increments $W(t)-W(s)\thicksim\mathcal{N}(0,t-s)$
          for $0\le s<t<\infty,$
    \item the sample trajectories for almost all $\omega\in\Omega$ the paths
          $t\to W(t,\omega)$ are almost surely continuous.
  \end{enumerate}
\end{defn}

\subsection{Ito Calculus}

The Ito integral is defined in a way that is similar to the Riemann integral.
The Ito integral is taken with respect to infinitesimal increments of a Brownian
motion, $\d{W}(t)$, which are random variables, while the Riemann integral
considers integration with respect to the predictable infinitesimal changes
$\d{t}$. The Ito integral is a random variable, while the Riemann integral is just
a real number. Despite this fact, there are several common properties and
relations between these two types of integrals.

\begin{defn} [Square Integrable]
  Let $f(t),t\ge0$ be a stochastic process with a.s. continuous paths
  adapted to the filtration $\mathcal{F}_{t}=\mathcal{F}^{W}(t)$, then
  we define the $\mathcal{M}^{2}$ space of processes as
  \begin{equation} 
    \mathcal{M}^{2} \defeq 
        \left\{
          f : [0,T] \times \Omega \to \mathbb{R} \mid
          f \text{ is adapted } \mid
          \mathbb{E} \left( \int_{0}^{T} f^{2}(t) dt \right) < \infty
        \right\}.
  \end{equation}
\end{defn}

\begin{defn} [Simple Process]
  A process $f\in\mathcal{M}^{2}$ is simple if we can find
  a partition $0=t_{0}<t_{1}<...<t_{n}=T$, and $\mathcal{F}_{t_{k}}$-measurable
  random variables $\xi_{k}$ with $\mathbb{E}(\xi_{k}^{2})<\infty$,
  $k=0,1,...,n-1$ such that
  \begin{equation}
    f(t,\omega)=\xi_{0}(\omega)\mathbf{1}_{\{0\}}(t)+\sum_{k=0}^{n-1}\xi_{k}(\omega)\mathbf{1}_{(t_{k},t_{k+1}]}(t),
  \end{equation}
 and we write $f\in\mathcal{S}^{2}$ to denote the class of all simple processes.
\end{defn}

\begin{defn} [The Ito integral]
  The Ito integral of a stochastic process $X(t)$ on a partition
  $[0,T]$, $0=t_{0}<t_{1}<\ldots<t_{n}=T$, with respect to the Brownian
  motion $W$ given by

  \begin{equation}
    \int_{0}^{T}X(t)\d{W(t)}\triangleq\lim_{n\to\infty}\sum_{j=0}^{n1}X(t_{j})\left(W(t_{j+1})-W(t_{j})\right).
  \end{equation}

\end{defn}

\begin{prop} 
  [
    {\cite{shreve1991}}
    Existence of the Ito integral
  ]
  
  The Ito stochastic integral 
  
  \begin{equation*}
    \int_{a}^{b} F_t \d{W}(t)
  \end{equation*}

  exists if the process $F_t = f(W(t),t)$ satisfies the following properties:

  \begin{enumerate}
    \item the paths $t \to F_t(\omega)$ are continuous on $[a,b]$ for any $\omega in \Omega$
    \item the process $F_t$ is non-anticipating for $t \in [a,b]$
    \item 
      \begin{equation*}
        \mathbb{E} \left( \int_{a}^{b} F^{2}_{t} \d{t} \right) < \infty.
      \end{equation*}
  \end{enumerate}
\end{prop}

\subsection{Itos Formula/Lemma}

Ito's formula is the analogue of the chain rule from elementary Calculus, we
present the multidimensional case.

\begin{thm} [Multidimensional Ito formula/Lemma]

  For a $n$-dimensional Ito process $X_{i}(\cdot)$ with $d$ independent Brownian motions and a
  matrix $[\beta_{ij}(t)]$, $i=1,\dots,n$ and $\nu=1,\dots,d$ of $\mathcal{L}_{[0,T]}^{2}$ processes

  \begin{equation}
    \d{X_{i}(t)} = \alpha_i(t) + \sum_{\nu=1}^d \beta_{i\nu}(t)\d{W_{\nu}(t)}
    \quad i=1,\dots,k,
  \end{equation}

  then for a given function $F:[0,\infty) \times \mathbf{R}^d \to \mathbf{R}$, of
  class $\mathcal{C}^2$ in the first variable and $\mathcal{C}^1$ in the others
  and write $Y(t)=F(t,X(t))$. Then $Y$ is an Ito process with

   \begin{gather}
    \begin{split}
    \d{Y(t)} &= F_{t}(t, X(t))\d{t} + \sum_{i=1}^n F_{x_{i}}(t,X(t)) \alpha_{i}(t)\d{t} \\
             & + \sum_{i=1}^n F_{x_{i}}(t,X(t)) \sum_{\nu=1}^n \beta_{i\nu}(t)\d{W_{\nu}(t)} \\
             & + \frac{1}{2} \sum_{\nu=1}^d \sum_{i=1}^n \sum_{j=1}^n
                  F_{x_{i}x_{j}}(t,X(t)) \beta_{i\nu}(t) \beta_{j\nu}(t)\d{t}.
    \end{split}
  \end{gather}
\end{thm}

\begin{cor} [Multidimensional Ito formula/Lemma where $n=1$]
  Where we are interested in the dynamics of a single process
  with $d$-dimensional Brownian motion, as above, with

  \begin{equation}
    \d{X(t)} = \alpha(t) + \sum_{\nu=1}^d \beta_{\nu}(t)\d{W_{\nu}(t)}
    \quad i=1,\dots,k,
  \end{equation}

  then for a given function $F:[0,\infty) \times \mathbf{R}^d \to \mathbf{R}$, of
  class $\mathcal{C}^2$ in the first variable and $\mathcal{C}^1$ in the others
  and write $Y(t)=F(t,X(t))$. Then $Y$ is an Ito process with

  \begin{gather}
    \begin{split}
    \d{Y(t)} &= F_{t}(t, X(t))\d{t} + F_{x}(t,X(t)) \alpha_{i}(t)\d{t} \\
             & + F_{x}(t,X(t)) \beta_{j}(t)\d{W_{j}(t)} \\
             & + \frac{1}{2} \sum_{\nu=1}^d F_{xx}(t,X(t))
                \beta^{2}_{\nu}(t)\d{t},\\
    \end{split}
  \end{gather}
\end{cor}

%%%%%%%%%%%%%%%%%%%%%%%%%%%%%%%%%%%%%%%%%%%%%%%%%%%%%%%%%%%%%%%%%%%%
\newpage
\section{Stochastic Portfolio Theory}

%%%%%%%%%%%%%%%%%%%%%%%%%%%%%%%%%%%%%%%%%%%%%%%%%%%%%%%%%%%%%%%%%%%%
\subsection{The Market Model}

We introduce the market model introduced by Fernholz (\cite{fernholz1999pgf} and
later in \cite{fernholz2009}) of stock price processes represented by continuous
semi-martingales, which is fairly standard in continuous-time financial theory
and investigated in detail in \cite{karatzas1998}.

A number of assumptions are made for clarity of expression, amongst these are:

\begin{itemize}
  \item the number of companies in the market is fixed, and companies do not 
        break up or merge,
  \item the number of shares of a company remains constant,
  \item trading is in continuous time,
  \item dividends are paid continuously,
  \item there are no transaction costs or taxes,
  \item fractional ownership of shares are allowed.
\end{itemize}

Although these assumptions are not based on the empirical facts of the market,
they are made to improve the clarity and simplicity of the exposition of the
theory. In most cases the theory can be generalised to include them. An
additional convention in SPT we adopt is to assume that each company has a
single share outstanding, so that the price of the stock is equivalent to it's
market capitalisation.

Our setting is a market $\market$, with $n$ stocks, and $d$-dimensional
independent Brownian motion $W(\cdot)$ (with $d \ge n$), defined on a
probability space $\probabilityspace$ as

\begin{gather}
  \label{eq:marketmodel}
  \begin{split}
    \d{B(t)} &= B(t)r(t)\d{t},  
      \quad \ranget, \\
    \d{X_{i}(t)} &= 
          X_{i}(t) 
          \left(
              b_{i}(t)dt + \sum_{\nu=1}^{d} \sigma_{i\nu}(t) dW_{\nu}(t)
          \right),
      \quad \rangei,
      \quad \ranget
  \end{split}
\end{gather}

where $W = \left\{ W(t)=(W_{1}(t),...,W_{d}(t)),\filtration{t},\ranget \right\}$
are the independent $d$dimensional Brownian motions, $r(\cdot)$ is the
interest rate process for the money-market, $B(0)=1$. The price process
$X_{i}(t)$ represents the price of the $i$th stock, where $X_{i}(0) = x_{i} > 0$
are the (strictly positive) initial values of the stock prices.

We also assume the $\filtration{}$progressively measurable $(n \times 1)$
process $b(\cdot)$ called the \textit{rates of return}, and the $(n \times d)$
process $\sigma_{i\nu}(t)$ of \textit{volatilities} satisfy the integrability
conditions: 

\begin{equation*}
  \int_{0}^{T} 
  \abs{r(t)} 
  \d{t} +
  \sum_{i=1}^{n} \int_{0}^{T} 
    \left( 
        \abs{b_{i}(t)} +
        \sum_{\nu=1}^{d} ( \sigma_{i\nu}(t)^2  ) 
        \right) \d{t} < \infty,
  \quad T \in [0, \infty),
  \quad \almostsurely.
  \end{equation*}


The stock price process (\ref{eq:marketmodel}) can be written using the notation 

\begin{equation}
  \label{eq:stockpriceprocessdiff}
    \frac{\d{X_{i}(t)}}{X_{i}(t)} = b_{i}(t)\d{t} + \sum_{\nu=1}^{d} \sigma_{i\nu}(t) dW_{\nu}(t),
  \quad \rangei
  \quad \ranget,
\end{equation}

where the first ratio defined as

\begin{equation*}
  \frac{\d{X_{i}(t)}}{X_{i}(t)} \defeq 
  \int_{0}^{t} b_{i}(s)\d{s} + 
  \int_{0}^{t} \sum_{\nu=1}^{d} \sigma_{i\nu}(s) dW_{\nu}(s),
  \quad \rangei,
\end{equation*}

is referred to as the \textit{instantaneous} return on the stock.

We now introduce the formal definition of a market ($\market$).

\begin{defn} 
[
  {\cite[Definition 2.2]{fernholz1999pgf}}
]
\label{def:market}

A market is a family $\market = \{X_{i},\dots,X_{n}\}$ of stocks, defined as in
(\ref{eq:marketmodel}), for which there is a number $\epsilon>0$ such that

\begin{equation}
  \label{eq:strongnondegeneracy}
  x \sigma(t) x^{T} \ge \epsilon \|x\|^{2}, 
  \quad x \in \mathbf{R}^{n}, 
  \quad \ranget,
  \quad \almostsurely.
\end{equation}

Here, $\sigma(t) = \{\sigma_{i\nu}(t)\}$ where $\rangei$ and $\nu=1,\dots,d$ is the
covariance matrix defined in (\ref{eq:marketmodel}) and $T$ denotes
transposition.

\end{defn}

The \textit{strong non-degeneracy} condition (\ref{eq:strongnondegeneracy}) is a 
common condition regarding market volatility, sometimes known as \textit{strict
non-degeneracy}, and can be found, for example, in \cite{shreve1991}. It 
expresses the requirement that the eigenvalues of the market covariation 
matrix be bounded away from zero. 

%%%%%%%%%%%%%%%%%%%%%%%%%%%%%%%%%%%%%%%%%%%%%%%%%%%%%%%%%%%%%%%%%%%%
\subsubsection{Stocks}

We follow the convention used by Fernholz and use a logarithmic representation
for stocks.

The logarithmic return (log return) of a financial asset is the change in the
natural logarithm of the asset's value. This is commonly referred to as the
"geometric" or "compound" return. Sometimes, the log return yields a clearer
picture of stock behaviour than is available from the usual arithmetic return
particularly in the case of certain stock portfolios
\cite{fernholz2007statistics}. A logarithmic representation is considered to be
more natural when considering long-term behaviour (see e.g.
\cite{fernholz1982}).

\begin{prop} [
  {\cite[Equation 1.5]{fernholz2009}} 
  Logarithmic Representation of Stock Price Process]
  \label{thm:logarithmicrepresentation}

  Let $X_{i}(\cdot)$ be stock price processes as defined under market $\market$,
  then

  \begin{equation}
    \label{eq:dlogX}
        \d{\log{X_{i}(t)}} =
          \gamma_{i}(t) \d{t} +
          \sum_{\nu=1}^{d} \sigma_{i\nu}(t) dW_{\nu}(t),
  \end{equation}

  where the process $\gamma_{i}(t)$, called the \textit{growth rate}, and the
  (non-negative definite matrix-valued) \textit{covariance process} $a_{ij}(t)$
  are defined as

  \begin{equation}
    \label{eq:gamma}
    \gamma_{i}(t)\defeq b_{i}(t) - \frac{1}{2}a_{ii}(t),
  \end{equation}

  \begin{gather}
    \label{eq:covarianceprocess}
    \begin{split}
      a_{ij}(t)
         \defeq \sum_{\nu=1}^{d}\sigma_{i\nu}(t)\sigma_{j\nu}(t).
%        & = \left( \sigma(t)\sigma'(t) \right)_{ij} \\
%        & = \frac{d}{dt}\left\langle \log X_{i},\log X_{j}\right\rangle(t)
    \end{split}
  \end{gather}

\end{prop}

\begin{proof}

  We use the multidimensional Ito's lemma with $F(t,x)=\log\{x\}$ for each stock
  independently, therefore we obtain that, for each $\rangei$,

  \begin{equation}
    F_{t}=0, \quad F_{x_{i}}=\frac{1}{x_{i}}, \quad 
    F_{x_{i}x_{i}}=-\frac{1}{x^2}, \quad F_{x_{i}x_{j}}=0, i \neq j.
  \end{equation}

  Then, using $\alpha=X_{i}(t)b_{i}(t)$ and $\beta=X_{i}(t)\sigma_{i\nu}(t)$, 
  from (\ref{eq:marketmodel}) we get

  \begin{gather}
    \begin{split}
    \d{\log{X_{i}(t)}} 
        =& \frac{1}{X_{i}(t)} X_{i}(t)b_{i}(t)\d{t} 
            + \frac{1}{X_{i}(t)} \sum_{\nu=1}^d X_{i}(t)\sigma_{i\nu}(t)\d{W_{\nu}(t)} \\ 
        &
            + \frac{1}{2} \sum_{\nu=1}^d \frac{1}{X_{i}^2(t)} X_{i}(t)\sigma_{i\nu}(t) X_{i}(t)\sigma_{i\nu}(t) \d{t},\\
        =& b_{i}(t)\d{t} 
            + \sum_{i\nu=1}^d \sigma_{i\nu}(t) \d{W_{\nu}(t)}
             - \frac{1}{2} \sum_{\nu=1}^d \sigma^2_{i\nu}(t) \d{t} \\
        =& \left( b_{i}(t) - \frac{1}{2} \sum_{\nu=1}^d \sigma^2_{i\nu}(t) \right) \d{t} 
            + \sum_{\nu=1}^d \sigma_{i\nu}(t) \d{W_{\nu}(t)}\\
        =& \left( b_{i}(t) - \frac{1}{2} \sum_{\nu=1}^d \sigma_{i\nu}(t) \sigma{i\nu} \right) \d{t} 
            + \sum_{\nu=1}^d \sigma_{i\nu}(t) \d{W_{\nu}(t)}\\
        =& \left( b_{i}(t) - \frac{1}{2} \sum_{\nu=1}^d a_{ii}(t) \right) \d{t} 
            + \sum_{\nu=1}^d \sigma_{i\nu}(t) \d{W_{\nu}(t)}.\\
        =& \gamma_{i}(t) \d{t} + \sum_{\nu=1}^d \sigma_{i\nu}(t) \d{W_{\nu}(t)}.
   \end{split}
  \end{gather}

\end{proof}

\newpage
%%%%%%%%%%%%%%%%%%%%%%%%%%%%%%%%%%%%%%%%%%%%%%%%%%%%%%%%%%%%%%%%%%%%
\subsubsection{Investment Strategies and Portfolios}

% TODO: Add more detail to Investment Strategies and Portfolios subsection intro 
In this section we introduce portfolios as a process of decisions an investor
takes that results in an allocation of capital to assets in the market.

\begin{defn} [Trading Strategy]
  \label{def:tradingstrategy}

  A \textit{trading strategy} is a progressively measurable process $h(\cdot)$
  that takes values in $\mathbb{R}^{n}$ with a wealth process $V^{w,h}(\cdot)$ 

  \begin{equation*}
    V^{w,h}(t) = \sum_{i=1}^{n} h_{i}(t) X_{i}(t) 
    \quad \ranget,
  \end{equation*}

  with $V^{w,h}(0)=w$ for $w > 0$. 

  We also assume a trading strategy $h(\cdot)$ satisfies the integrability condition

  \begin{equation*}
    \sum_{i=1}^{n} \int_{0}^{T} 
    \left(
    \abs{(h_{i}(t)b_{i}(t)} + h_{i}^2(t)a_{ii}(t)
      \right) \d{t} < \infty
    \quad \almostsurely.
  \end{equation*}

\end{defn}

\begin{defn} [Self Financing Condition]
  \label{def:selffinancingcondition}  
  A strategy $h(\cdot)$ is called \textit{self financing} if 

  \begin{equation}
    \d{V^{w,h}(t)} = \sum_{i=1}^{n} h_{i}(t) \d{X_{i}(t)}.
  \end{equation}

  \comment{to me above condition looks incorrect.}

\end{defn}

\begin{defn} [Portfolio]
  \label{def:portfolio}

  A portfolio is a progressively measurable process $\pi(\cdot)$ uniformly
  bounded in $(t,\omega)$, where $\pi_{i}(t)$ represents the proportion of wealth
  invested in stock $i$ at time $t$, with values in the set $\triangle^{n}$,
  defined as 

  \begin{equation*}
    \triangle^{n} \defeq 
    \left\{
          (\pi_{1}, \dots, \pi_{n}) \in \mathbb{R}^{n} 
          \mid
          \sum_{i=1}^{n} \pi_{i} = 1
    \right\}.
  \end{equation*}

  A negative value for $\pi_{i}(t)$ indicates a short position, we also define a
  \textit{long only} portfolio $\pi(\cdot)$ as a portfolio where $\pi_{i}(t) \ge
  0$ $\forall \rangei$. We introduce the notation for this set as

  \begin{equation*}
    \triangle_{+}^{n} \defeq 
    \left\{
          (\pi_{1}, \dots, \pi_{n}) \in \triangle^{n} 
          \mid
          \pi_{1} \ge 0, \dots, \pi_{n} \ge 0
          \mid
          \sum_{i=1}^{n} \pi_{i} = 1
    \right\}.
  \end{equation*}

\end{defn}

Consider the wealth process $\V(\cdot)$ of a portfolio $\pi(\cdot)$, then by
definition, the weights $h_{i}(\cdot)$ of the trading strategy corresponding to
the portfolio are related as follows:

\begin{gather}
  \begin{split}
    V^{w,h}(t) =\sum_{i=1}^{n} h_i(t)X_{i}(t) = \sum_{i=1}^{n}
                    \pi_{i}(t)V^{w,\pi}(t), \\
  \end{split}
\end{gather}

Therefore

\begin{gather}
  \begin{split}
    \label{eq:wealthinvestedbyportfolio}
    h_i(t) &= \frac{\pi_{i}(t)V^{w,\pi}(t)}{X_{i}(t)},
    \quad \rangei.
  \end{split}
\end{gather}

Using this relationship, and by applying the self financing condition we can
investigate the dynamics of the wealth process, and formulate the following
proposition that relates the return of the portfolio $\frac{\d{\V(t)}}{\V(t)}$
to the weighted average of the returns of the individual stocks, given by the
portfolio weights $\pi_{i}(\cdot)$.

\begin{prop} [{\cite[Equation 2.2]{fernholz2009}} Wealth Process of a Portfolio]

  The wealthprocess $V^{w,\pi}(\cdot)$ of a portfolio $\pi(\cdot)$
  with initial wealth $w > 0$ satisfies the stochastic differential equation

  \begin{gather}
    \label{eq:arithmeticreturnofportfolio}
    \begin{split}
      \frac{\d{V^{w,\pi}(t)}}{V^{w,\pi}(t)} 
        &= \sum_{i=1}^{n} \pi_{i}(t) \frac{\d{X_{i}(t)}}{X_{i}(t)} \\
        &= b_{\pi}(t)\d{t} + \sum_{\nu=1}^{d} \sigma_{\pi\nu}(t) \d{W_{\nu}(t)}
    \end{split}
  \end{gather}

  where we define the process $b(\cdot)$, called the \textit{rate-of-return}, and the
  volatility coefficients $\sigma_{\pi\nu}(\cdot)$ of the portfolio $\pi(\cdot)$ 
  as

  \begin{equation}
    \label{eq:wealthprocessrateofreturn}
    b_{\pi}(t) \defeq \sum_{i=1}^{n} \pi_{i}(t) b_{i}(t)
  \end{equation}

  \begin{equation}
    \label{eq:wealthprocessvolatility}
    \sigma_{\pi\nu}(\cdot) \defeq \sum_{i=1}^{n} \pi_{i}(t) \sigma_{i\nu}(t)
    \quad \nu=1,\dots,d.
  \end{equation}

\end{prop}

\begin{proof}

  From the self financing condition (Definition \ref{def:selffinancingcondition}) 
  and (\ref{eq:wealthinvestedbyportfolio}) we have

  \begin{gather*}
    \begin{split}
      \d{V^{w,\pi}(t)} 
      &= \sum_{i=1}^{n} h_{i}(t) \d{X_{i}(t)} \\
      &= \sum_{i=1}^{n} \left(  \frac{\pi_{i}(t)V^{w,\pi}(t)}{X_{i}(t)} \right) \d{X_{i}(t)} \\
      &= V^{w,\pi}(t) \sum_{i=1}^{n} \pi_{i}(t) \frac{\d{X_{i}(t)}}{X_{i}(t)} \\
      \frac{\d{V^{w,\pi}(t)}}{V^{w,\pi}(t)} 
      &= \sum_{i=1}^{n} \pi_{i}(t) \frac{\d{X_{i}(t)}}{X_{i}(t)} \\
    \end{split}
  \end{gather*}

   which completes the proof for the first part of (\ref{eq:arithmeticreturnofportfolio}). 
  We prove the second equality by using (\ref{eq:stockpriceprocessdiff}), so that
  
  \begin{gather*}
    \begin{split}
      \frac{\d{V^{w,\pi}(t)}}{V^{w,\pi}(t)}
          & = \sum_{i=1}^{n} \pi_{i}(t) 
          \left(
            b_{i}(t)\d{t} + \sum_{\nu=1}^{d} \sigma_{i\nu}(t) \d{W_{\nu}}(t)
          \right) \\
          & = \sum_{i=1}^{n} \pi_{i}(t) b_{i}(t)\d{t} + 
              \sum_{i=1}^{n} \sum_{\nu=1}^{d} \pi_{i}(t) \sigma_{i\nu}(t) \d{W_{\nu}}(t) \\
          & = \sum_{i=1}^{n} \pi_{i}(t) b_{i}(t)\d{t} + 
              \sum_{\nu=1}^{d} \sum_{i=1}^{n} \pi_{i}(t) \sigma_{i\nu}(t) \d{W_{\nu}}(t) \\
          & = b_{\pi}(t)\d{t} + 
              \sum_{\nu=1}^{d} \sum_{i=1}^{n} \pi_{i}(t) \sigma_{i\nu}(t) \d{W_{\nu}}(t) \\
          & = b_{\pi}(t)\d{t} + 
              \sum_{\nu=1}^{d} \sigma_{\pi\nu}(t) \d{W_{\nu}}(t).
    \end{split}
  \end{gather*}

\end{proof}

Equation (\ref{eq:arithmeticreturnofportfolio}) shows that a portfolio's
arithmetic return is similar to that of it's underlying stocks, with an 
arithmetic return drift rate and a volatility term. The portfolio's arithmetic
return drift rate is the weighted average of its stocks' arithmetic return drift
rates, and the volatility is the weighted average of its stocks volatility
process $\sigma_{i\nu}(\cdot)$.

Since SPT is interested in the long term behaviour of portfolios, solutions of
(\ref{eq:arithmeticreturnofportfolio}) are of great interest, we therefore
provide a proof of the solution to the stochastic differential equation
describing the arithmetic return of a portfolio. 

\begin{prop} 
  [
    {\cite[Equation 1.10]{fernholz2009}} 
    Strong Solution of Arithmetic Return of Portfolio
  ]
  \label{prop:solutionofarithmeticreturnofportfolio}

  Let $\pi(\cdot)$ be a portfolio, then the solution of
  (\ref{eq:arithmeticreturnofportfolio}) is

  \begin{equation}
    \label{eq:solutionofarithmeticreturnofportfoliodiff}
    \d\log{\V(t)} =  
        \gamma_{\pi}(t) \d{t} +
        \sum_{\nu=1}^{d} \sigma_{\pi\nu}(t) \d{W_{\nu}(t)} ,
  \end{equation}

  or equivalently, as

  \begin{equation}
    \label{eq:solutionofarithmeticreturnofportfolio}
    \V(t) = w \exp{ 
      \left(
        \int_{0}^{t} \gamma_{\pi}(u) \d{u} +
        \sum_{\nu=1}^{d} \int_{0}^{t} \sigma_{\pi\nu}(u) \d{W_{\nu}(u)}
      \right)},
  \quad \ranget
  \end{equation}

  where we define the process $\gamma_{\pi}(\cdot)$, called the \textit{growth
  rate} and the process $\gamma_{\pi}^{*}(\cdot)$, called the \textit{excess
  growth rate} of the portfolio $\pi(\cdot)$ as follows

  \begin{equation}
    \label{eq:portfoliogrowthrate}
    \gamma_{\pi}(t) \defeq 
      \sum_{i=1}^{n} \pi_{i}(t)\gamma_{i}(t) + 
      \gamma_{\pi}^{*}(t)
  \end{equation}

  \begin{equation}
    \label{eq:portfolioexcessgrowthrate}
    \gamma_{\pi}^{*}(t) \defeq \frac{1}{2} 
        \left(
          \sum_{i=1}^{n} \pi_{i}(t)a_{ii}(t) -
          \sum_{i=1}^{n} \sum_{j=1}^{n} \pi_{i}(t)a_{ij}(t)\pi_{j}(t)
        \right)
  \end{equation}

\end{prop}
\newcommand{\Valpha}{\V(t)b_{\pi}(t)}
\newcommand{\Vbeta}{\V(t)\sigma_{\pi\nu}(t)}
\newcommand{\VFt}{0}
\newcommand{\VFx}{\frac{1}{\V(t)}}
\newcommand{\VFxx}{-\frac{1}{(\V(t))^2}}
\begin{proof}

  As before, we use the multidimensional Ito's lemma with $F(t,x)=\log\{x\}$ therefore we obtain that

  \begin{equation}
    F_{t}=0, \quad F_{x}=\frac{1}{x}, \quad F_{xx}=-\frac{1}{x^2}.
  \end{equation}

  Then, using $\alpha=\Valpha$ and $\beta=\Vbeta$, from (\ref{eq:arithmeticreturnofportfolio}) we get

  \begin{gather*}
    \begin{split}
      \d{Y(t)} &= F_{t}(t, X(t))\d{t} + F_{x}(t,X(t)) \alpha(t)\d{t} 
         + F_{x}(t,X(t)) \sum_{\nu=1}^d \beta_{\nu}(t)\d{W_{\nu}(t)} \\
         &+ \frac{1}{2} \sum_{\nu=1}^d F_{xx}(t,X(t)) \beta_{\nu}(t)^{2}(t)\d{t} \\
     \d{\log{\V(t)}} 
         &= \VFx \Valpha \d{t} + \VFx \sum_{\nu=1}^d \Vbeta \d{W_{\nu}(t)} \\
            &+ \frac{1}{2} \sum_{\nu=1}^d \VFxx \left( \Vbeta \right)^{2} \d{t} \\
         &= b_{\pi}(t) \d{t} 
            + \sum_{\nu=1}^d \sigma_{\pi\nu}(t)  \d{W_{\nu}(t)} 
             - \frac{1}{2} \sum_{\nu=1}^d \left( \sigma_{\pi\nu}(t)\sigma_{\pi\nu}(t) \right) \d{t}. \\
    \end{split}
  \end{gather*}

  Recalling the definition of $b_{i\pi}(\cdot)$ (\ref{eq:wealthprocessrateofreturn})
  , $\sigma_{\pi\nu}(\cdot)$ (\ref{eq:wealthprocessvolatility}) and
  $a_{ij}(\cdot)$ (\ref{eq:covarianceprocess}) gives:

  \begin{equation*}
    b_{\pi}(t) \defeq \sum_{i=1}^{n} \pi_{i}(t) b_{i}(t),
  \end{equation*}

  \begin{equation*}
    \sigma_{\pi\nu}(\cdot) \defeq \sum_{i=1}^{n} \pi_{i}(t) \sigma_{i\nu}(t),
    \quad \nu=1,\dots,d,
  \end{equation*}

  \begin{gather*}
    \begin{split}
      a_{ij}(t) \defeq \sum_{\nu=1}^{d}\sigma_{i\nu}(t)\sigma_{j\nu}(t).
    \end{split}
  \end{gather*}

  We then have
  
  \begin{gather*}
    \begin{split}
      \d{\log{\V(t)}} 
        &= \sum_{i=1}^{n} \pi_{i}(t) b_{i}(t) \d{t} 
            + \sum_{\nu=1}^d \sigma_{\pi\nu}(t) \d{W_{\nu}(t)} 
            - \frac{1}{2} \sum_{\nu=1}^{d}
                \left( \sum_{i=1}^{n} \pi_{i}(t) \sigma_{i\nu}(t) \right) 
                \left( \sum_{j=1}^{n} \pi_{j}(t) \sigma_{j\nu}(t) \right) 
              \d{t} \\
        &= \sum_{i=1}^{n} \pi_{i}(t) b_{i}(t) \d{t} 
            + \sum_{\nu=1}^d \sigma_{\pi\nu}(t) \d{W_{\nu}(t)} 
            - \frac{1}{2} \left( \sum_{i=1}^{n} \sum_{j=1}^{n} \pi_{i}(t) 
              \left( 
                \sum_{\nu=1}^{d} \sigma_{i\nu}(t) \sigma_{j\nu}(t) 
              \right) \pi_{j}(t) \right) \d{t} \\
        &= \sum_{i=1}^{n} \pi_{i}(t) b_{i}(t) \d{t} 
            + \sum_{\nu=1}^d \sigma_{\pi\nu}(t) \d{W_{\nu}(t)} 
            - \frac{1}{2} \sum_{i=1}^{n} \sum_{j=1}^{n} \pi_{i}(t) a_{ij}(t) \pi_{j}(t) \d{t}. \\
   \end{split}
  \end{gather*}

  From (\ref{eq:gamma}) we have

  \begin{gather*}
    \begin{split}
      \gamma_{i}(t) & \defeq b_{i}(t) - \frac{1}{2}a_{ii}(t) \\
       \text{rearranging so } b_{i}(t) &= \gamma_{i}(t) + \frac{1}{2}a_{ii}(t). \\
    \end{split}
  \end{gather*}

  Substituting in gives:

  \begin{gather*}
    \begin{split}
      \d{\log{\V(t)}} 
        & =
            \sum_{i=1}^{n} \pi_{i}(t) \left( 
                \gamma_{i}(t)\d{t} + \frac{1}{2}a_{ii}(t) 
            \right)\d{t}
           - \frac{1}{2} \sum_{i=1}^{n} \sum_{j=1}^{n} \pi_{i}(t)a_{ij}(t)\pi_{j}(t)\d{t}
          + \sum_{\nu=1}^{d} \sigma_{\pi\nu}(t) \d{W_{\nu}(t)} \\
        & =
            \sum_{i=1}^{n} \pi_{i}(t)\gamma_{i}(t)\d{t} +
                \frac{1}{2} \sum_{i=1}^{n} \pi_{i}(t)a_{ii}(t) \d{t}
                - \frac{1}{2} \sum_{i=1}^{n} \sum_{j=1}^{n} \pi_{i}(t)a_{ij}(t)\pi_{j}(t)\d{t}
          + \sum_{\nu=1}^{d} \sigma_{\pi\nu}(t) \d{W_{\nu}(t)} \\
         & =
          \left(
            \sum_{i=1}^{n} \pi_{i}(t)\gamma_{i}(t) +
            \frac{1}{2}
              \left(
                \sum_{i=1}^{n} \pi_{i}(t)a_{ii}(t) -
                \sum_{i=1}^{n} \sum_{j=1}^{n} \pi_{i}(t)a_{ij}(t)\pi_{j}(t)
              \right) 
          \right)\d{t} +
          \sum_{\nu=1}^{d} \sigma_{\pi\nu}(t) \d{W_{\nu}(t)} \\
         & =
          \left(
            \sum_{i=1}^{n} \pi_{i}(t)\gamma_{i}(t) +
            \gamma_{\pi}^{*}(t)
          \right)\d{t} +
          \sum_{\nu=1}^{d} \sigma_{\pi\nu}(t) \d{W_{\nu}(t)} \\
         &=
          \gamma_{\pi}(t) \d{t} +
          \sum_{\nu=1}^{d} \sigma_{\pi\nu}(t) \d{W_{\nu}(t)}.
   \end{split}
  \end{gather*}

\end{proof}

Note that the excess growth rate process $\gamma_{\pi}^{*}(\cdot)$ is the
difference between the weighted sum of the individual stock variances and the
overall portfolio variance, and has therefore been interpreted as the returns due
to diversification. It was proved in \cite{fernholz1999diversity} that for
portfolios with non-negative weights, the excess growth rate is non-negative,
and is positive unless the portfolio consists of a single stock.

\begin{prop} 
  [
    {\cite[Equation 1.15]{fernholz2009}}
    Portfolio Wealth Process
  ]
  \label{prop:dlogV}

  Let $\V(\cdot)$ be the wealth process of a portfolio $\pi(\cdot)$ in market
  $\market$, then the wealth process satisfies the stochastic differential equation

  \begin{equation}
      \d{\log \V(t)} = \gamma_{\pi}^{*}(t)\d{t} + \sum_{i=1}^{n} \pi_{i}(t) \d{\log{X_{i}(t)}}.
  \end{equation}

\end{prop}

\begin{proof}
  We start by recalling the definitions of the following processes over the last
  few sections, specifically the portfolio growth rate $\gamma_{\pi}(\cdot)$
  (\ref{eq:portfoliogrowthrate}) and the volatility of the wealth process of a
  portfolio $\sigma_{\pi\nu}(\cdot)$ (\ref{eq:wealthprocessvolatility}), 
  
  \begin{equation}
    \gamma_{\pi}(t) \defeq 
      \sum_{i=1}^{n} \pi_{i}(t)\gamma_{i}(t) + \gamma_{\pi}^{*}(t),
  \end{equation}

  and

  \begin{equation}
    \sigma_{\pi\nu}(\cdot) \defeq \sum_{i=1}^{n} \pi_{i}(t) \sigma_{i\nu}(t)
    \quad \nu=1,\dots,d.
  \end{equation}

  From the result of Proposition
\ref{prop:solutionofarithmeticreturnofportfolio}, we have 

  \begin{gather}
    \begin{split}
      \label{eq:dlogVstep1}
      \d\log{\V(t)} 
      &=  
        \gamma_{\pi}(t) \d{t} +
        \sum_{\nu=1}^{d} \sigma_{\pi\nu}(t) \d{W_{\nu}(t)} \\
      &=
        \gamma_{\pi}(t) \d{t} +
        \sum_{\nu=1}^{d} \sum_{i=1}^{n} \pi_{i}(t) \sigma_{i\nu}(t) \d{W_{\nu}(t)} \\
    \end{split}
  \end{gather}

  From (\ref{eq:dlogX}), we have

  \begin{equation}
        \d{\log{X_{i}(t)}} =
          \gamma_{i}(t) \d{t} +
          \sum_{\nu=1}^{d} \sigma_{i\nu}(t) \d{W_{\nu}}(t).
  \end{equation}

  then multiplying by $\sum_{i=1}^{n} \pi_{i}(t)$ we have 
 
  \begin{gather*}
    \begin{split}
       \sum_{i=1}^{n} \pi_{i}(t) \d{\log{X_{i}(t)}} 
        &=
          \sum_{i=1}^{n} \pi_{i}(t) \gamma_{i}(t) \d{t} +
          \sum_{i=1}^{n} \pi_{i}(t) \sum_{\nu=1}^{d} \sigma_{i\nu}(t) \d{W_{\nu}}(t) \\
        &=
          \sum_{i=1}^{n} \pi_{i}(t) \gamma_{i}(t) \d{t} +
          \sum_{\nu=1}^{d} \sum_{i=1}^{n} \pi_{i}(t) \sigma_{i\nu}(t) \d{W_{\nu}}(t) \\
    \end{split}
  \end{gather*}

  Rearranging we get 

  \begin{gather*}
    \begin{split}
       \sum_{\nu=1}^{d} \sum_{i=1}^{n} 
            \pi_{i}(t) \sigma_{i\nu}(t) \d{W_{\nu}}(t) &=
       \sum_{i=1}^{n} \pi_{i}(t) \d{\log{X_{i}(t)}} -
       \sum_{i=1}^{n} \pi_{i}(t) \gamma_{i}(t) \d{t}. \\
    \end{split}
  \end{gather*}

  To complete the proof we can then substitute back into (\ref{eq:dlogVstep1}),
  and use the  definition of the portfolio growth rate $\gamma_{\pi}(\cdot)$
  (\ref{eq:portfoliogrowthrate}) to get 

  \begin{gather}
    \begin{split}
      \d\log{\V(t)} 
      &=
        \gamma_{\pi}(t) \d{t} +
        \sum_{i=1}^{n} \pi_{i}(t) \d{\log{X_{i}(t)}} -
        \sum_{i=1}^{n} \pi_{i}(t) \gamma_{i}(t) \d{t}. \\
      &=
        \left( 
            \sum_{i=1}^{n} \pi_{i}(t)\gamma_{i}(t) + \gamma_{\pi}^{*}(t) 
        \right) \d{t} +
        \sum_{i=1}^{n} \pi_{i}(t) \d{\log{X_{i}(t)}} -
        \sum_{i=1}^{n} \pi_{i}(t) \gamma_{i}(t) \d{t}. \\
       &=
        \left( 
            \sum_{i=1}^{n} \pi_{i}(t)\gamma_{i}(t) 
            + \gamma_{\pi}^{*}(t) 
            - \sum_{i=1}^{n} \pi_{i}(t) \gamma_{i}(t)
        \right) \d{t} +
        \sum_{i=1}^{n} \pi_{i}(t) \d{\log{X_{i}(t)}} \\
       &=
        \gamma_{\pi}^{*}(t)\d{t} +
        \sum_{i=1}^{n} \pi_{i}(t) \d{\log{X_{i}(t)}}.
    \end{split}
  \end{gather}

\end{proof}

To simplify notation slightly, going forward we write $V^{\pi}(\cdot) \defeq V^{1,\pi}$ for initial wealth $w=1$.

\newpage
%%%%%%%%%%%%%%%%%%%%%%%%%%%%%%%%%%%%%%%%%%%%%%%%%%%%%%%%%%%%%%%%%%%%
\subsection{The Market Portfolio}

We now introduce the \textit{Market Portfolio}, which is perhaps the most
important portfolio in SPT. The main objective under SPT is to relate the 
performance of two different portfolios, so naturally one needs a 
benchmark portfolio to perform these comparisons against. The Market Portfolio
fulfils this requirement in SPT.

The portfolio buys at time $t=0$ the same number of shares in all stocks, and
holds on to them (recalling our assumption that each stock has just one share
outstanding, so that capitalization and stock price are the same).

\begin{defn} [Market Portfolio] 
  \label{def:marketportfolio}

  We define the process $\mu(\cdot)$ as a portfolio that invests the proportion
  $\mu_{i}(t)$ of current wealth in the $i$th asset at all times

  \begin{equation} 
    \label{eq:marketportfolio} 
      \mu_{i}(t) \defeq \frac{X_{i}(t)}{X(t)}, 
    \quad 
    \rangei 
  \end{equation}

  where we define $X(t)$, as the total market capitalisation at time $t$ by

  \begin{equation} 
    \label{eq:totalmarketcapitalisation} 
      X(t) \defeq \sum_{i=1}^{n} X_{i}(t) 
  \end{equation}

\end{defn}

\begin{prop} [Arithmetic Return of Market Portfolio]

  The arithmetic return of the Market Portfolio satisfies

  \begin{equation} 
    \label{eq:arithmeticreturnofmarketportfolio}
      \frac{d{\Vmu}(t)}{\Vmu(t)} = \frac{\d{X(t)}}{X(t)}.
  \end{equation}

\end{prop}


\begin{proof}

  From (\ref{eq:arithmeticreturnofportfolio})

  \begin{gather} 
    \begin{split} 
      \frac{d{\Vmu}(t)}{\Vmu}(t) 
      & =\sum_{i=1}^{n} \mu_{i}(t) \frac{\d{X_{i}(t)}}{X_{i}(t)} \\ 
      & = \sum_{i=1}^{n} \frac{X_{i}(t)}{X(t)} \frac{\d{X_{i}(t)}}{X_{i}(t)} \\ 
      & = \sum_{i=1}^{n} \frac{\d{X_{i}(t)}}{X(t)} \\ 
      & = \frac{\d{X(t)}}{X(t)} \\
    \end{split} 
  \end{gather}

\end{proof}

From (Proposition \ref{prop:solutionofarithmeticreturnofportfolio}) we can
write, for the market portfolio $\mu(\cdot)$

\begin{equation} 
  \label{eq:dlogV}
  \d{\log{\Vmu}(t)} =  \gamma_{\mu}(t)\d{t} + 
        \sum_{\nu=1}^{d} \sigma_{\mu\nu}(t) \d{W_{\nu}(t)} 
\end{equation}

\begin{prop} [Dynamics of the Market Weights]
  \label{prop:dynamicsofmarketweights}

  The weight $\mu_{i}(\cdot)$ process of the Market Portfolio satisfies 

  \begin{equation} 
    \label{eq:eqmarketportfolioweights}
    \d{\log{\mu_{i}(t)}} = (\gamma_{i}(t) - \gamma_{\mu}(t))\d{t} +
        \sum_{\nu=1}^{d} \left( \sigma_{i\nu}(t) - \sigma_{\mu\nu}(t) \right) \d{W_{\nu}(t)} 
  \end{equation}

\end{prop}

\begin{proof}
  
  From (\ref{eq:marketportfolio}) we can write the market weights as a function
  of the wealth process, as the wealth of the market portfolio is the total market
  capitalisation (again assuming starting wealth $w=1$), and then using the results 
  (\ref{eq:dlogX}) and (\ref{eq:dlogV}) we get

  \begin{gather} 
    \begin{split} 
      \d{\log{\mu_{i}(t)}} 
        &= \d{\log{ \left( \frac{ X_{i}(t) }{ \Vmu(t) } \right)}}  \\
        &= \d{ \log{X_{i}(t)} } - \d{ \log{\Vmu(t)} } \\ 
        &= 
            \left(
              \gamma_{i}(t) \d{t} + \sum_{\nu=1}^{d} \sigma_{i\nu}(t) \d{W_{\nu}(t)}
            \right) -
            \left(
              \gamma_{\mu}(t)\d{t} + \sum_{\nu=1}^{d} \sigma_{\mu\nu}(t) \d{W_{\nu}(t)} 
            \right) \\
        &=
            (\gamma_{i}(t) - \gamma_{\mu}(t))\d{t} +
              \sum_{\nu=1}^{d} 
              \left( 
                \sigma_{i\nu}(t) - \sigma_{\mu\nu}(t) 
              \right) \d{W_{\nu}(t)}. 
    \end{split} 
  \end{gather}

\end{proof}

The relative market weights evolve as

  \begin{gather} 
    \label{eq:relativemarketweights}
    \begin{split} 
      \frac{\d{\mu_{i}(t)}}{\mu_{i}(t)} &=
        \left(
         \gamma_{i}(t) - \gamma_{i}(t) + \frac{1}{2} 
            \sum_{\nu=1}^{d}
            \left(
              \sigma_{i\nu}(t) - \sigma_{\mu\nu}(t)
            \right) ^ 2
        \right) \d{t} + 
        \sum_{\nu=1}^{d}
        \left(
          \sigma_{i\nu}(t) - \sigma_{\mu\nu}(t)
        \right) \d{W_{\nu}(t)}  \\
        &=
        \left(
         \gamma_{i}(t) - \gamma_{i}(t) + \frac{1}{2} \tau_{ii}^{\mu}(t)
        \right) \d{t} + 
        \sum_{\nu=1}^{d}
        \left(
          \sigma_{i\nu}(t) - \sigma_{\mu\nu}(t)
        \right) \d{W_{\nu}(t)}  \\
    \end{split}
  \end{gather}

where we define $\tau_{ij}^{\pi}$, the matrix-valued \textit{covariance process}
of the stocks relative to the portfolio $\pi(\cdot)$ as

\begin{equation}
  \label{eq:eqtau}
  \tau_{ij}^{\pi}(t) \defeq
            \sum_{\nu=1}^{d}
            \left(
              \sigma_{i\nu}(t) - \sigma_{\mu\nu}(t)
            \right) 
            \left(
              \sigma_{j\nu}(t) - \sigma_{\mu\nu}(t)
            \right) 
\end{equation}

Expanding this out and using the definition of $a_{ij}$
from (\ref{eq:covarianceprocess}), we can also write this as

  \begin{gather}
    \begin{split}
      \tau_{ij}^{\pi}(t) 
        &=
            \sum_{\nu=1}^{d}
            \left(
              \sigma_{i\nu}(t) - \sigma_{\mu\nu}(t)
            \right) 
            \left(
              \sigma_{j\nu}(t) - \sigma_{\mu\nu}(t)
            \right) \\
        &=
            \sum_{\nu=1}^{d}
            \left(
              \sigma_{i\nu}(t)\sigma_{j\nu}(t) - \sigma_{i\nu}(t)\sigma_{\mu\nu}(t) -
              \sigma_{\mu\nu}(t)\sigma_{j\nu}(t) + \sigma_{\mu\nu}(t)\sigma_{\mu\nu}(t)
            \right) \\
        &= a_{ij}(t) - a_{\pi i}(t) - a_{\pi j}(t) + a_{\pi\pi}(t)
   \end{split}
  \end{gather}

Where we define

  \begin{equation}
      a_{\pi i}(t) \defeq \sum_{j=1}^{n} \pi_{j}(t)a_{ij}(t), \quad 
      a_{\pi \pi}(t) \defeq \sum_{i=1}^{n} \sum_{j=1}^{n} \pi_{i}(t) \pi_{j}(t) a_{ij}(t).
  \end{equation}

\newpage
%%%%%%%%%%%%%%%%%%%%%%%%%%%%%%%%%%%%%%%%%%%%%%%%%%%%%%%%%%%%%%%%%%%%
\section{Relative Return and Arbitrage}

Having defined stocks and portfolios, and in particular the important market
portfolios, this section examines the relationship between portfolios, in 
a \textit{relative} sense. We prove an important proposition that relates
the relative return of a portfolio against the market portfolio in terms
of purely the weights of the market portfolio.

Relative arbitrage in SPT and the classical formulation of arbitrage are
different, we decribe briefly these differences before moving on to the formal
definitions of relative returns and relative arbitrage. Classical arbitrage is
measured versus cash, which in the formulation of SPT can be considered to be a
constant, positive process. The relative arbitrage we present here is generally
versus the market portfolio $\mu(\cdot)$ (\ref{eq:marketportfolio}), and this
means that the market portfolio replaces cash as the benchmark against which the
relative arbitrage is measured. 

In general, all types of arbitrage have a boundedness restriction to prevent
"doubling" strategies. For classical arbitrage, the value of the trading
strategy that creates the arbitrage must be bounded from below relative to cash.
In relative arbitrage under SPT, this bound is again performed relative to the
market. In general, there is no bound on the market value, so one bound does not
yield the other one, and vice versa. Hence, these two types of arbitrage can be
incompatible.

\subsection{Relative Return of two Portfolios}

We begin our investigation into relative return and arbitrage by examining the
relative return between two portfolios under our model $\model$.

\begin{lem} 
  [
    {\cite[Lemma 3.2]{fernholz2009}}
    Relative Return of Two Portfolio
  ]
  \label{lem:relativereturnoftwoportfolios}

  For any two arbitrary portfolios $\pi(\cdot)$ and $\rho(\cdot)$, we have the
  following dynamics

  \begin{equation}    
    \label{eq:rrdynamics} 
      \d{\log{ \left( \frac{ V^{\pi}(t) }{V^{\rho}(t) } \right) } } = 
        \gamma_{\pi}^{*}(t)\d{t} + 
         \sum_{i=1}^{n} \pi_{i}(t) 
            \d{\log{ \left( \frac{ X_{i}(t) }{ V^{\rho}(t)} \right) }}. 
  \end{equation}

  In particular, for a portfolio $\pi(\cdot)$ relative to the market portfolio
  $\mu(\cdot)$ we have

  \begin{gather} 
    \label{eq:rrdynamics2} 
    \begin{split} 
      \d{ \log{ \left( \frac{V^{\pi}(t) }{ V^{\mu}(t) } \right) } } 
      &= \gamma_{\pi}^{*}(t)\d{t} + 
            \sum_{i=1}^{n} \pi_{i}(t)  \d{ \log{\mu_{i}(t)} } \\ 
      &= (\gamma_{\pi}^{*}(t) - \gamma_{\mu}^{*}(t)) \d{t} +
            \sum_{i=1}^{n} (\pi_{i}(t) - \mu_{i}(t)) \d{\log{\mu_{i}(t)} }.
    \end{split} 
  \end{gather}

\end{lem}

\begin{proof}

  From Proposition \ref{prop:dlogV}, for two arbitrary portfolios $\pi(\cdot)$
  and $\rho(\cdot)$, we have

  \begin{gather} 
    \begin{split} 
      \d{\log V^{\pi}(t)} &= 
          \gamma_{\pi}^{*}(t)\d{t} + \sum_{i=1}^{n} \pi_{i}(t)\d{\log{X_{i}(t)}}, \\
      \d{\log V^{\rho}(t)} &= 
          \gamma_{\rho}^{*}(t)\d{t} + \sum_{i=1}^{n} \rho_{i}(t) \d{\log{X_{i}(t)}}.
     \end{split} 
  \end{gather}

  Therefore, recalling $\sum_{i=1}^{n} \pi_{i}(t) \defeq 1$, we can write (\ref{eq:rrdynamics}) as

  \begin{gather*} 
    \begin{split} 
      \d{\log{ \left( \frac{ V^{\pi}(t) }{V^{\rho}(t) } \right) }} 
      &=  \d{\log V^{\pi}(t)} - \d{\log V^{\rho}(t)} \\
      &=  \left( 
              \gamma_{\pi}^{*}(t)\d{t} + \sum_{i=1}^{n} \pi_{i}(t) \d{\log{X_{i}(t)}} 
          \right)
          - \d{\log V^{\rho}(t)} 
          \left(
            \sum_{i=1}^{n} \pi_{i}(t) 
          \right) \\
      &=  \gamma_{\pi}^{*}(t)\d{t} + 
              \sum_{i=1}^{n} \pi_{i}(t) 
          \left( 
                  \d{\log{X_{i}(t)}} - \d{\log V^{\rho}(t)} 
          \right) \\
      &=  \gamma_{\pi}^{*}(t)\d{t} + \sum_{i=1}^{n} \pi_{i}(t) 
           \d{\log{ \left( \frac{ X_{i}(t) }{ V^{\rho}(t)} \right) }}.
    \end{split} 
  \end{gather*}

  The first equation of (\ref{eq:rrdynamics2}) is a simple extension of
  (\ref{eq:rrdynamics}) noting again (from proof of Proposition
  \ref{prop:dynamicsofmarketweights}) and with $\rho(\cdot)=\mu(\cdot)$ that,

  \begin{equation*} 
      \d{\log{\mu_{i}(t)}} 
        = \d{\log{ \left( \frac{ X_{i}(t) }{ \Vmu(t) } \right)}},
  \end{equation*}

  then

  \begin{gather} 
    \label{eq:eqrrdynamicseq1}
    \begin{split} 
      \d{\log{ \left( \frac{ V^{\pi}(t) }{V^{\rho}(t) } \right) }} 
      &=  \gamma_{\pi}^{*}(t)\d{t} + \sum_{i=1}^{n}\pi_{i}(t)\d{\log{\mu_{i}(t)}},
    \end{split} 
  \end{gather}

  which completes the proof of the first equality in (\ref{eq:rrdynamics2}). We now 
  prove the second equality in (\ref{eq:rrdynamics2}), starting from
  (\ref{eq:eqrrdynamicseq1}), so that

  \begin{gather*} 
    \begin{split} 
     \d{ \log{ \left( \frac{V^{\pi}(t) }{ V^{\mu}(t) } \right) } } 
      &= \gamma_{\pi}^{*}(t)\d{t} + 
            \sum_{i=1}^{n} \pi_{i}(t)  \d{ \log{\mu_{i}(t)} } \\ 
    \end{split} 
  \end{gather*}

  Using the dynamics of the market portfolio weights (\ref{eq:eqmarketportfolioweights}), we have

  \begin{gather*} 
    \begin{split} 
      \sum_{i=1}^{n} \mu_{i}(t) \d{\log{\mu_{i}(t)}} 
    &= 
      \sum_{i=1}^{n} \mu_{i}(t) \left( \gamma_{i}(t) - \gamma_{\mu}(t) \right)\d{t} \\
    &= 
      -\gamma^{*}_{\mu}(t)\d{t}.
    \end{split}
  \end{gather*}

  Therefore (working backwards), we have

  \begin{gather*} 
    \begin{split} 
     \d{ \log{ \left( \frac{V^{\pi}(t) }{ V^{\mu}(t) } \right) } } 
      &= (\gamma_{\pi}^{*}(t) - \gamma_{\mu}^{*}(t)) \d{t} +
            \sum_{i=1}^{n} (\pi_{i}(t) - \mu_{i}(t)) \d{\log{\mu_{i}(t)} } \\
      &= (\gamma_{\pi}^{*}(t) - \gamma_{\mu}^{*}(t)) \d{t} +
            \sum_{i=1}^{n} \pi_{i}(t) \d{\log{\mu_{i}(t)} } - 
            \sum_{i=1}^{n} \mu_{i}(t) \d{\log{\mu_{i}(t)} } \\
      &= (\gamma_{\pi}^{*}(t) - \gamma_{\mu}^{*}(t)) \d{t} +
            \sum_{i=1}^{n} \pi_{i}(t) \d{\log{\mu_{i}(t)} } 
            + \gamma^{*}_{\mu}(t)\d{t} \\
      &= \gamma_{\pi}^{*}(t) \d{t} + \sum_{i=1}^{n} \pi_{i}(t) \d{\log{\mu_{i}(t)} } 
    \end{split} 
  \end{gather*}

  which completes the proof.

\end{proof}

The consequence of this proposition is that we can represent the relative 
return of any portfolio versus the market portfolio in terms of the changes 
in the market weights.

\begin{thm} [Relative Return Formula] For any portfolio $\pi(\cdot)$ we have
  \begin{equation*} 
    \d{ \log{ \left( \frac{ V^{\pi}(t) }{ V^{\mu}(t) } \right) }} = 
    \sum_{i=1}^{n} \frac{\pi_{i}(t)}{\mu_{i}(t)} \d{\mu_{i}(t)} -
    \frac{1}{2} \left( 
        \sum_{i=1}^{n} \sum_{j=1}^{n} \pi_{i}(t) \pi_{i}(t) \tau_{ij}^{\mu}(t)
    \right) \d{t}. 
  \end{equation*} 
\end{thm}

%\begin{proof} 
%  TODO: Prove Relative Return Formula
%\end{proof}

%%%%%%%%%%%%%%%%%%%%%%%%%%%%%%%%%%%%%%%%%%%%%%%%%%%%%%%%%%%%%%%%%%%%
\subsection{Relative Arbitrage}

One of the objectives of SPT is to examine \textit{relative arbitrage}
relationships between investment strategies. Of major importance then is to
define mathematically what is meant by arbitrage. We provide short definitions
of both \textit{strong} and \textit{weak} arbitrage.

\begin{defn} [Relative Arbitrage]
  \label{def:defrelativearbitrage}

  Let $h(\cdot)$ and $k(\cdot)$ be trading strategies. Then $h(\cdot)$ is called
  a relative arbitrage over $[0,T]$ with respect to $k(\cdot)$ is their associated
  wealth processes satisfy

  \begin{equation}
    V^{h}(T)\ge V^{k}(T)\;\;\text{a.s},\;\;\;\mathbb{P}(V^{h}(T)>V^{k}(T))>0.
  \end{equation}

\end{defn}

\begin{defn} [Strong Relative Arbitrage]
  \label{def:defstrongrelativearbitrage}

  A strong relative arbitrage has the following stronger inequality

  \begin{equation}
    \mathbb{P}(V^{h}(T)>V^{k}(T))=1.
  \end{equation}

\end{defn}

%%%%%%%%%%%%%%%%%%%%%%%%%%%%%%%%%%%%%%%%%%%%%%%%%%%%%%%%%%%%%%%%%%%%
\subsection{Numeraire Property}

In a link with the Benchmark approach to Mathematical finance \cite{platen2006},
when building relative arbitrage strategies, SPT relies on benchmarking a
portfolio against another portfolio, called the numeraire portfolio. The
\textit{numeraire property} is a property of the excess growth rate
(\ref{eq:portfolioexcessgrowthrate}) that shows that it is numeraire invariant,
as it shows that we may simply replace the covariance matrix in
(\ref{eq:portfolioexcessgrowthrate}) by the matrix of covariances relative to
any portfolio. When the numeraire proposed is the canonical benchmark market
portfolio defined in Definition \ref{def:marketportfolio}, this property
demonstrates the fact that the excess growth rate of a portfolio does not depend
on the choice of numeraire portfolio.

\begin{lem} [
    {\cite[Lemma 2.2.2]{vervuurt2015}}
    Numeraire Invariance Property
    ]
  For any two arbitrary portfolios $\pi(\cdot)$ and $\rho(\cdot)$, we have the
  following property 

  \begin{equation} 
    \label{eq:numeraireinvarianceproperty}
    \gamma_{\pi}^{*}(t) = \frac{1}{2} 
        \left(
          \sum_{i=1}^{n} \pi_{i}(t)\tau_{ii}^{\rho}(t) - 
          \sum_{i=1}^{n} \sum_{j=1}^{n} \pi_{i}(t)\pi_{j}(t)\tau_{ii}^{\rho}(t) 
        \right)
  \end{equation}

  and in particular, for a long-only portfolio $\pi(\cdot)$ we have that

  \begin{equation} 
    \label{eq:numeraireinvariancepropertylongonly}
    \gamma_{\pi}^{*}(t) = \frac{1}{2} 
          \sum_{i=1}^{n} \pi_{i}(t)\tau_{ii}^{\pi}(t) 
     \ge 0. 
  \end{equation}
 
\end{lem}

%\begin{proof}
%
%  From definition of $\tau(\cdot)$ (\ref{eq:eqtau}) we have
%
%  \begin{equation} 
%    \gamma_{\pi}^{*}(t) = \frac{1}{2} 
%          \sum_{i=1}^{n} \pi_{i}(t)\tau_{ii}^{\pi}(t) 
%     \ge 0. 
%  \end{equation}
%
%\end{proof}

\newpage
%%%%%%%%%%%%%%%%%%%%%%%%%%%%%%%%%%%%%%%%%%%%%%%%%%%%%%%%%%%%%%%%%%%%
\section{Functionally Generated Portfolios}

Functionally generated portfolios were used by Fernholz
\cite{fernholz1999diversity} to study market diversity and to establish
conditions under which arbitrage will exist in markets. In
\cite{fernholz1999pgf}, Fernholz presented a general discussion of these
portfolios and the mathematical functions which generate them. The returns on
these portfolios were shown to be related to the return on the market portfolio
by a stochastic differential equation. Under appropriate conditions, this
equation can be used to establish a dominance, or relative arbitrage
relationship between a functionally generated portfolio and the market
portfolio.

In addition to being used for studying theoretical questions such as market
diversity and arbitrage, these portfolios have been used for actual equity
investments. An institutional investment product based on a functionally
generated portfolio of the stocks in the S\&P 500 Index was introduced in 1996 by
Intech (now part of Janus Capital). The theory has also been used by Fernholz
\cite{fernholz1999diversity} to investigate the well known size factor, the
historical tendency of smaller stocks to have higher returns than larger stocks.

%%%%%%%%%%%%%%%%%%%%%%%%%%%%%%%%%%%%%%%%%%%%%%%%%%%%%%%%%%%%%%%%%%%%
\subsection{Generating Function}

In this section we shall show that certain real-valued functions of the market
weights can be used to generate portfolios, we introduce in this section the
properties of these functions and the portfolios they generate. 

SPT deals with a number of different types of portfolio generating functions
that mostly fall into two categories: smooth functions of the market weights, 
and smooth functions of the ranked market weights. 

Those portfolio generating functions that are smooth functions of the market
weights can be used to create portfolios with returns that satisfy almost sure
relationships relative to the market portfolio, and, hence, can be applied to
situations in which arbitrage might be possible. Those functions that are smooth
functions of the ranked market weights can be used to analyze the role of
company size in portfolio behaviour.

We restrict our investigation to generating functions that are smooth functions
of the market weights, as these are used to generate diversity weighted
portfolios that we investigate in later sections.

\newcommand{\G}[1]{\mathbf{G}(#1)}
\newcommand{\Gmu}{\G{\mu(t)}}
\newcommand{\g}{\mathfrak{g}(t)}

\begin{defn} [Generating Function] 
  \label{def:generatingfunction}

  Let $U \subset \triangle^{n}$ be a given open set, then the function
  $\G{\cdot} \in \mathcal{C}^{2}(U,(0,\infty))$ is a generating function for the
  portfolio $\pi(\cdot)$ if $\G{\cdot}$ is such that $x\to
  x_{i}D_{i}\log\G{x}$ is bounded on $U$, and if there exists a
  measurable, adapted process $\mathfrak{g}(\cdot)$ such that 

  \begin{equation}
    \d{ \log \left( \frac{V^{\pi}(t)}{V^{\mu}(t)} \right) } = 
    \d{\log{\Gmu}} + \g 
    \quad \ranget
    \quad \almostsurely.
  \end{equation}

\end{defn}

\begin{prop} [
  {\cite[Proposition 2.3.2]{vervuurt2015}}
  Weights of portfolio generated by portfolio generating function
  ]
  \label{prop:generatingfunction}

  Let a function $\G{}$ as in Definition \ref{def:generatingfunction}
  generate the portfolio $\pi(\cdot)$, then for each of the portfolio weights
  $\pi_{i}(t)$, $\rangei$

  \begin{equation}
    \label{eq:portfoliogeneratedbyG}
    \pi_{i}(t) = 
      \left( 
        D_{i}\log{\Gmu} + 1 -
          \sum_{j=1}^{n} \mu_{j}(t) D_{j} \log{\Gmu}
      \right) \cdot \mu_{i}(t).
  \end{equation}

\end{prop}

\begin{proof}
  First we verify $\pi(\cdot)$ is a valid portfolio, i.e. fully invested.

   \begin{gather}
    \begin{split}
      \sum_{i=1}^{n} \pi_{i}(t) 
      &= \sum_{i=1}^{n} 
        \left( 
          D_{i}\log{\Gmu} + 1 - 
            \sum_{j=1}^{n} \mu_{j}(t)D_{j}\log{\Gmu}
        \right) \mu_{i}(t) \\
      &= 
        \sum_{i=1}^{n} \mu_{i}(t) D_{i}\log{\Gmu} + 
        \sum_{i=1}^{n} \mu_{i}(t)
        \left( 
          1 - \sum_{j=1}^{n} \mu_{j}(t)D_{j}\log{\Gmu}
        \right) \\
       &= 
        \sum_{i=1}^{n} \mu_{i}(t) D_{i}\log{\Gmu} + 
        \sum_{i=1}^{n} \mu_{i}(t) -
        \sum_{i=1}^{n} \mu_{i}(t) \sum_{j=1}^{n} \mu_{j}(t)D_{j}\log{\Gmu} \\
       &= 1,
    \end{split}
  \end{gather}

  since by definition $\sum_{i=1}^{n} \mu_{i}(t) = 1$.

  The remainder of the proof is performed by formulating a lemma shown below that
  proves the reverse direction of Proposition \ref{prop:generatingfunction}, (as
  presented in \cite{fernholz2009} and \cite{vervuurt2015}). A proof of the
  reverse direction was performed in \cite{fernholz1999pgf}. 

\end{proof}

%%%%%%%%%%%%%%%%%%%%%%%%%%%%%%%%%%%%%%%%%%%%%%%%%%%%%%%%%%%%%%%%%%%%
\subsection{Fernholz's Master Equation}

% TODO: Add more detail to intro to Master Equation

\begin{lem} 
  [
    {\cite[Theorem 3.1]{\fernholz1999pgf}}
    Fernholz's Master Equation
  ]

  \label{lem:lemmamasterequation}

  For a portfolio $\pi(\cdot)$ satisfying \ref{prop:generatingfunction}, then the
  generating function $\G{}$ generates the portfolio $\pi(\cdot)$, i.e.

  \begin{equation}
    \label{eq:masterequation}
    \log \left( \frac{V^{\pi}(t)}{V^{\mu}(t)} \right) = 
    \log \left( \frac{\G{\mu(T)}}{\G{\mu(0)}} \right) + 
      \int_{0}^{T} \g \d{t}
    \quad \almostsurely
  \end{equation}

  where 

  \begin{equation}
    \label{eq:eqdefg}
    \g \triangleq \frac{-1}{2 \Gmu}
        \sum_{i=1}^{n} \sum_{j=1}^{n} D_{ij}^{2} \Gmu 
        \mu_{i}(t) \mu_{j}(t)
        \tau_{ij}^{\mu}(t)
  \end{equation}

  is called the \textit{drift process} of the portfolio $\pi(\cdot)$.

\end{lem}

\begin{proof}

For the first part of the proof we derive a useful expression for the LHS of
(\ref{eq:masterequation}, we have from Lemma
\ref{lem:relativereturnoftwoportfolios} equation (\ref{eq:rrdynamics2}), that

  \begin{gather} 
    \begin{split} 
      \d{ \log{ \left( \frac{V^{\pi}(t) }{ V^{\mu}(t) } \right) } } 
        &= \gamma_{\pi}^{*}(t)\d{t} + 
            \sum_{i=1}^{n} \pi_{i}(t)  \d{ \log{\mu_{i}(t)} }, \\ 
    \end{split} 
  \end{gather}

  and expressions (\ref{eq:eqmarketportfolioweights}) and (\ref{eq:relativemarketweights})
  and applying the numeraire invariance property
  (\ref{eq:numeraireinvarianceproperty}) to get
    
  \begin{gather} 
    \label{eq:masterstep1}
    \begin{split} 
      \d{ \log{ \left( \frac{V^{\pi}(t) }{ V^{\mu}(t) } \right) } } 
        &= 
            \sum_{i=1}^{n} \frac{\pi_{i}(t)}{\mu_{i}(t)} \d{\mu_{i}(t)} 
            - \frac{1}{2} \sum_{i=1}^{n} \sum_{j=1}^{n} 
            \pi_{i}(t)\pi_{j}(t)\tau_{ij}^{\mu}\d{t}.
    \end{split} 
  \end{gather}

  Now we derive the dynamics of $\log{\G{\mu(\cdot)}}$. Note the relation

  \begin{gather} 
    \begin{split} 
      D_{ij}^2\log{\Gmu} &= \frac{D_{ij}^2\log{\Gmu}}{\Gmu} - D_{i}\Gmu \cdot D_{j}\Gmu.
    \end{split} 
  \end{gather}

  To simplify notation we set

  \begin{gather} 
    \begin{split} 
        g_{i}(t) = D_{i}\log{\Gmu} \\
        N(t) = 1 - \sum_{j=1}^{n} \mu_{j}(t)g_{j}(t),
    \end{split} 
  \end{gather}

then from (\ref{eq:eqtau}) we have that
 
  \begin{gather} 
    \label{eq:x}
    \begin{split} 
      \d{\log{\G{\mu(\cdot)}}} 
        &=  \sum_{i=1}^{n} g_{i}(t) \d{\mu_{i}}(t) +
            \frac{1}{2} \sum_{i=1}^{n} \sum_{j=1}^{n} D_{ij}^2\log{\Gmu} 
            \d{\langle \mu_{i}, \mu_{j} \rangle}(t) \\
        &=  \sum_{i=1}^{n} g_{i}(t) \d{\mu_{i}}(t) +
            \frac{1}{2} \sum_{i=1}^{n} \sum_{j=1}^{n} 
            \left(
              \frac{D_{ij}^2\log{\Gmu}}{\Gmu} - g_{i}(t)g_{j}(t) 
            \right)
            \mu_{i} \mu_{j} \tau_{ij}^{\mu} \d{t}. 
   \end{split} 
  \end{gather}

With the temporary notation the equation (\ref{eq:portfoliogeneratedbyG})
becomes

  \begin{equation}
    \pi_{i}(t) = (g_{i}(t) + N(t))\mu_{i}(t),
  \end{equation} 

and we use this to derive an expression for the first term in
(\ref{eq:masterstep1}) as

  \begin{gather} 
    \label{eq:masterterm1}
    \begin{split} 
        \sum_{i=1}^{n} \frac{\pi_{i}(t)}{\mu_{i}(t)} \d{\mu_{i}(t)} 
        &= \sum_{i=1}^{n} g_{i}\d{\mu_{i}(t)} + N(t)
            \d{ \left( \sum_{i=1}^{n} \mu_{i}(t) \right) } \\
        &= \sum_{i=1}^{n} g_{i}\d{\mu_{i}(t)}
   \end{split} 
  \end{gather}

and then the second term of (\ref{eq:masterstep1}) as

  \begin{gather} 
    \label{eq:masterterm2}
    \begin{split} 
      \sum_{i=1}^{n} \sum_{j=1}^{n} \pi_{i}(t)\pi_{j}(t)\tau_{ij}^{\mu} 
      &=
        \sum_{i=1}^{n} \sum_{j=1}^{n} 
          (g_{i}(t) + N(t))(g_{j}(t) + N(t)) 
         \mu_{i} \mu_{j}(t) \tau_{ij}^{\mu}(t) \\
      &=
        \sum_{i=1}^{n} \sum_{j=1}^{n} 
          g_{i}(t)g_{j}(t)\mu_{i}\mu_{j}(t)\tau_{ij}^{\mu}(t)
   \end{split} 
  \end{gather}

since from the definition of $\tau(\cdot)$ (\ref{eq:eqtau}), we have

\newcommand{\apii}{\sum_{j=1}^{n}\pi_{j}(t)a_{ij}(t)}
\newcommand{\apipi}{\sum_{i=1}^{n}\sum_{j=1}^{n}\pi_{i}(t)\pi_{j}(t)a_{ij}(t)}

  \begin{gather} 
    \begin{split} 
    \sum_{j=1}^{n} \pi_{j}(t) \tau_{ij}^{\mu} (t)
    &= 
    \sum_{j=1}^{n} \pi_{j}(t) a_{ij}(t) - \apii -
    \sum_{j=1}^{n} \pi_{j}(t) a_{ij}(t) + \apipi \\
    &= 0. \\
   \end{split} 
  \end{gather}

Hence from (\ref{eq:masterterm1}) and (\ref{eq:masterterm2}), we can write
equation (\ref{eq:masterstep1}) as

  \begin{gather} 
    \begin{split} 
      \d{ \log{ \left( \frac{V^{\pi}(t) }{ V^{\mu}(t) } \right) } } 
        &= 
            \sum_{i=1}^{n} \frac{\pi_{i}(t)}{\mu_{i}(t)} \d{\mu_{i}(t)} 
            - \frac{1}{2} \sum_{i=1}^{n} \sum_{j=1}^{n} 
            \pi_{i}(t)\pi_{j}(t)\tau_{ij}^{\mu}\d{t} \\
        &= \sum_{i=1}^{n} g_{i}\d{\mu_{i}(t)}  -\frac{1}{2} 
            \sum_{i=1}^{n} \sum_{j=1}^{n} 
            g_{i}(t)g_{j}(t)\mu_{i}(t)\mu_{j}(t)\tau_{ij}^{\mu}(t) \d{t} \\
        &= \sum_{i=1}^{n}  D_{i}\log{\Gmu} d{\mu_{i}(t)} -
            \frac{1}{2}\sum_{i=1}^{n} \sum_{j=1}^{n} 
             D_{ij}^2 \log{\Gmu} D_{j}\log{\Gmu} 
            \mu_{i}\mu_{j}(t)\tau_{ij}^{\mu}(t)
    \end{split} 
  \end{gather}

  and the result follows by comparison with (\ref{eq:x}) and our definition of
  $\g$ in (\ref{eq:eqdefg}).

\end{proof}

The formulation of the Master Equation is of great importance in SPT, as it
allows us to relate the observed properties of markets (and thus conditions on
the behaviour of certain processes over time) to the relative performance of a
portfolio compared to the market portfolio. 

The method of constructing relative arbitrage opportunities in SPT, is to choose
suitable generating function $\G$, by noting that the first term on the RHS of
(\ref{eq:masterequation}) can be bounded from below. Furthermore, the volatility
processes only appear in the drift process $\g$, and the drift processes do not
appear at all in (\ref{eq:masterequation}).

In the paper \cite{pal2016geometry} it is proven that for the class of
portfolios that depend only on the current market capitalisations, then a slight
generalisation of functionally generated portfolios is the only class that can
lead to a relative arbitrage.

\newpage
%%%%%%%%%%%%%%%%%%%%%%%%%%%%%%%%%%%%%%%%%%%%%%%%%%%%%%%%%%%%%%%%%%%%
\section{Diverse Models}

We now turn our attention to a consequence of the theory that has preceded this
section, by looking at diverse models. These are portfolios generated by a
generating function (Definition \ref{def:generatingfunction}), which can be
shown using the master equation (Lemma \ref{lem:lemmamasterequation}) to
generate a relative arbitrage over the market portfolio.

\subsection{Definition of Diversity}

An equity market is called \textit{diverse} if no single stock is ever allowed
to dominate the entire market in terms of relative capitalization. In this
section we give a formal definition of market diversity and then characterize
market diversity in terms of the excess growth rate.

\begin{defn} 
  [
    {\cite[Definition 3.2]{fernholz1999diversity}}
    Diversity Condition
  ]
  \label{def:defnD}

  A market $\market$ is diverse on $[0,T]$ if the largest weight in the market
  portfolio (denoted $\mu_{max}(t)$) never accounts for the entire market
  capitalisation. More formally for some $\delta\in(0,1)$ such that

  \begin{equation}
    \label{eq:eqD}
      \mu_{max}(t) < 1 - \delta
    \quad \ranget,
    \quad \almostsurely.
  \end{equation}

\end{defn}

\begin{defn}
  [
    {\cite[Equation 5.2]{fernholz2009}}
    Weak Diversity Condition
  ]
  A market $\market$ is called weakly diverse on $[0,T]$ if the diversity
  condition (Definition \ref{def:defnD}) holds on average. More formally 
  we say if for some $\delta\in(0,1)$ we have

  \begin{equation}
    \label{eq:eqWD}
      \frac{1}{T}\int_{0}^{T}\mu_{max}(t) < 1 - \delta
      \quad \forall t\in[0,T],
      \quad \almostsurely.
  \end{equation}
\end{defn}

The diversity condition (\ref{eq:eqD}) asserts that a market is diverse if no
single company's capitalization can take up more than a certain proportion of
the entire market, and is weakly diverse (\ref{eq:eqWD}) if the diversity
condition holds on average over history. These are fairly weak empirical
requirements, and are observable. They have been shown to be broadly consistent
 with the actual behavior of equity markets in developed markets, especially 
in the presence of antitrust and competition laws.

We introduce a proposition from \cite{fernholz1999diversity} that relates the
excess growth rate of the market to diversity, which can be shown to preclude
diversity if all stocks in the market have the same {\cite[Corollary
3.1]{fernholz1999diversity}} or constant {\cite[Corollary
3.2]{fernholz1999diversity}}.

\begin{prop}
[
  {\cite[Proposition 3.1]{fernholz1999diversity}}
  Excess growth rate and diversity
]
  A market $\market$ is diverse if and only if there is a $\delta > 0$ such that

  \begin{equation}
      \gamma^{*}(t) \ge \delta
      \quad \ranget,
      \quad \almostsurely.
  \end{equation}

\end{prop}

\newpage
%%%%%%%%%%%%%%%%%%%%%%%%%%%%%%%%%%%%%%%%%%%%%%%%%%%%%%%%%%%%%%%%%%%%
\subsection{Diversity-Weighted Portfolio (DWP)}

Fernholz \cite{fernholz1999diversity} and \cite{fernholz2009} show that if the
model $\market$ of (\ref{eq:marketmodel}) is weakly diverse over the
time-interval $[0,T]$, and if the strong non-degeneracy condition
(\ref{eq:strongnondegeneracy}) holds, then $\market$ contains arbitrage
opportunities relative to the market portfolio, at least for sufficiently long
time horizons $T \in (0,\infty)$. They provide an example of a portfolio, called
the diversity weighted portfolio (DWP) that can be show to provide such relative
arbitrages over the market portfolio.

The diversity weighted portfolio $\pi^{(p)}(\cdot)$, is a functionally generated
portfolio with arbitrary parameter $p\in\mathbb{R}$ whose weights are defined 
in terms of the market portfolio $\mu(\cdot)$ (\ref{def:marketportfolio}) by

\begin{equation}
 \label{eq:DWP}
 \pi_{i}^{(p)}(t) \triangleq 
      \frac{ \left(\mu_{i}(t)\right)^{p} }
           {\sum_{j=1}^{n}\left(\mu_{j}(t)\right)^{p}},
   \quad i=1,...,n.
\end{equation}

\newcommand{\Gp}[1]{\mathbf{G}_{p}(#1)}

Fernholz \cite[Equation 7.3]{fernholz2009} shows that the portfolio
$\pi^{(p)}(\cdot)$ is generated by the $\mathcal{C}^2$, symmetric, concave
function $\Gp{x}$, which can be interpreted as a \textit{measure of diversity}

\begin{equation}
  \label{eq:eqDWPG}
  \Gp{x}:x\to\left(\sum_{i=1}^{n}x_{i}^{p}\right)^{\frac{1}{p}}
  \quad
\end{equation}

In definition of $\Gp{x}$, symmetry ensures that all stocks are treated in the
same manner, and the concavity implies that transferring capital from a larger
company to a smaller one will increase the measure. Thus it meets Fernholz's definition
of a \textit{measure of diversity} \cite[Definition 4.3]{fernholz1999pgf}.

An application of Itos lemma to (\ref{eq:eqDWPG}) leads to the expression

\begin{equation}
  \log \left( \frac{V^{\pi^{(p)}}(T)}{V^{\mu}(T)} \right) = 
  \log \left( \frac{\Gp{\mu(T)}}{\Gp{\mu(0)}} \right) + 
    (1-p)\int_{0}^{T}\gamma_{\pi^{(p)}}^{*}\d{t},
  \quad \almostsurely
\end{equation}

for the wealth $V^{\pi^{(p)}}(\cdot)$ of the diversity weighted portfolio
$\pi^{(p)}$ of (\ref{eq:DWP}).

The bounds of the function (\ref{eq:eqDWPG}), where minimum diversity occurs when the
entire market is concentrated in one stock, and maximum diversity when all
stocks have the same capitalization, are

\begin{equation}
  1 = \sum_{i=1}^{n} \mu_{i}(t) \le \sum_{i=1}^{n} (\mu_{i}(t))^p
    = \left( \Gp{\mu(t)} \right)^p \le n^{1-p},
\end{equation}

so therefore

\begin{equation}
  \log \left( \frac{\Gp{\mu(T)}}{\Gp{\mu(0)}} \right) \ge 
       -\frac{1-p}{p} \log{n}. 
\end{equation}

This shows therefore that $\frac{V^{\pi^{(p)}}(\cdot)}{V^{\mu}(\cdot)}$ is
bounded from below by the constant $m^{1-(1-p)/p}$, so that the requirement of
relative arbitrage (Definition \ref{def:defstrongrelativearbitrage}) is met.
\newpage
%%%%%%%%%%%%%%%%%%%%%%%%%%%%%%%%%%%%%%%%%%%%%%%%%%%%%%%%%%%%%%%%%%%%
\section{Empirical Analysis}

% TODO: Add details on empirical analysis

Using a subset of a proprietary dataset we perform an analysis of the market and
diversity weighted portfolio introduced in the sections above. The dataset
covers the 4-year period 2014-2017 for European equities that compromise the
Eurostoxx 600 Index. We ensure there is no survivorship bias by using the
composition of the index on each day of the simulation. If an equity falls out
of the index, the weight of the equity is set to zero in the portfolios. The
returns are corrected for stock splits and other corporate actions such as
dividends and buybacks.

\subsection{Diversity of Eurostoxx 600 Index}

In Figure \ref{fig:figdiversity} we see the cumulative changes in the diversity
of the Eurostoxx 600 index over the period 2014-2017, as measured by
$\Gp{\cdot}$ with $p=0.5$, normalised so the average over the whole range is 0.

The chart replicates \cite[Figure 1]{fernholz2009}, and shows the cumulative 
changes in diversity due to capital gains and losses, for the Eurostoxx 600
index. Fernholz performs a similar analysis in \cite[Chapter 7]{fernholz2002}
for S\&P500 index. The values used in are normalized so that the average over the analysis
is zero. Similarly to Fernholz's observation we do see indications that 
diversity is mean-reverting, however with only 4 years of data, it is hard to
see the longer term trends of around 10 to 20 years. 

Fernholz \cite{fernholz2009} notes that for extreme lows for diversity seem to
accompany bubbles, and we can see that over the bull-market of 2016-2017,
diversity is low.

\begin{figure}[!ht]
  \label{fig:figdiversity}
  \caption{Diversity of Euro Stoxx  using the diversity measure $\Gp{\cdot}$
  with $p=0.5$ normalised so average over the range is $0$.}
  \includegraphics[width=350pt,height=250pt]{diversity.pdf}
\end{figure}

\newpage
\subsection{Simple Return Analysis}

We begin our analysis by looking at the performance of the diversity weighted
portfolios as formulated by the SPT framework. Initially we do not account for
any market frictions such as transaction and market impact costs.

Using the formula for the weights of the DWP (\ref{eq:DWP}) we generated the
weights corresponding to different values of $p \in [-1,1]$. We explicitly
expand the original formulation of the DWP with $p > 0$ \cite{fernholz2005} with
the negative values proposed by Vervuurt \cite{vervuurt2015}.

Using corporate action adjusted one day forward returns we can then compute the
value process of each strategy. The daily returns are accumulated through the
year and presented as an annualised return. This is the theoretical frictionless
return as predicted by the SPT framework.

\begin{figure}[!ht]
  \label{fig:figsimpleperformance}
  \caption{Theoretical performance of the market portfolio (red) vs diversity weighted
portfolios generated using the diversity measure $\Gp{\cdot}$ for a range of
  values of $p \in [-1.0, 1.0]$.}
  \includegraphics[width=400pt,height=350pt]{performance.pdf}
\end{figure}

\begin{table}[!ht]
  \caption{Annualised Return of market vs diversity weighted portfolios}
  \begin{tabular}{lrrrrrr}
    \toprule
    \bf  & \bf 2014 & \bf 2015 & \bf 2016 & \bf 2017 \bf Total \\
    \midrule
    market & \color{red}5.2 & 0.4 & \color{red}-0.8 & 4.9 & \color{red}-0.7 \\
    dwp(1.0) & \color{red}-4.4 & 1.6 & 1.2 & 4.9 & 0\\
    dwp(0.7) & \color{red}-4.4 & 1.4 & 1.1 & 4.9 & 0\\
    dwp(0.4) & \color{red}-4.4 & 1.2 & 0.9 & 4.9 & 0\\
    dwp(0.2) & \color{red}-4.4 & 1.1 & 0.7 & 5.0 & 0\\
    dwp(+0.4) & \color{red}4.7 & 0.6 & 0.0 & 5.0 & 0\\
    dwp(+0.7) & \color{red}4.9 & 0.5 & \color{red}-0.4 & 5.0 & 0\\
    dwp(+1.0) & \color{red}5.2 & 0.4 & \color{red}-0.8 & 4.9 & 0\\
    \bottomrule
  \end{tabular}
\end{table}

From \ref{fig:figsimpleperformance} we can see that 

\afterpage{
    \clearpage    
    \thispagestyle{empty}
      \begin{landscape}
        \centering

\begin{table}[!ht]
    \caption{table}{Table caption}
\begin{tabular}{lrrrrrrrr}
\toprule
  & market & dwp.m1.0.1 & dwp.m0.7.0.1 & dwp.m0.4.0.1 & dwp.m0.2.0.1 & dwp.0.4.0.1 & dwp.0.7.0.1 & dwp.1.0.1\\
  \midrule
  Semi Deviation & 0.0067 & 0.0075 & 0.0070 & 0.0069 & 0.0068 & 0.0067 & 0.0067 & 0.0067\\
  Gain Deviation & 0.0060 & 0.0061 & 0.0060 & 0.0059 & 0.0059 & 0.0060 & 0.0060 & 0.0060\\
  Loss Deviation & 0.0069 & 0.0083 & 0.0075 & 0.0073 & 0.0072 & 0.0071 & 0.0070 & 0.0069\\
  Downside Deviation (MAR=210\%) & 0.0118 & 0.0124 & 0.0120 & 0.0118 & 0.0118 & 0.0118 & 0.0118 & 0.0118\\
  Downside Deviation (Rf=0\%) & 0.0066 & 0.0074 & 0.0069 & 0.0067 & 0.0067 & 0.0066 & 0.0066 & 0.0066\\
  Downside Deviation (0\%) & 0.0066 & 0.0074 & 0.0069 & 0.0067 & 0.0067 & 0.0066 & 0.0066 & 0.0066\\
  Maximum Drawdown & 0.2321 & 0.3392 & 0.2697 & 0.2400 & 0.2314 & 0.2281 & 0.2306 & 0.2321\\
  Historical VaR (95\%) & 0.0144 & -0.0158 & -0.0153 & -0.0153 & -0.0150 & -0.0146 & -0.0146 & -0.0144\\
  Historical ES (95\%) & 0.0220 & -0.0244 & -0.0227 & -0.0221 & -0.0220 & -0.0220 & -0.0220 & -0.0220\\
  Modified VaR (95\%) & 0.0152 & -0.0173 & -0.0161 & -0.0157 & -0.0155 & -0.0153 & -0.0153 & -0.0152\\
  Modified ES (95\%) & 0.0257 & -0.0406 & -0.0337 & -0.0315 & -0.0306 & -0.0281 & -0.0269 & -0.0257\\
  \bottomrule
\end{tabular}
\end{table}

  \end{landscape}
\clearpage
}

\newpage
\subsection{Simulation with Market Impact}

We extend the analysis to show how the exact same portfolios would perform if
run through a simulator that accounts for market impact using a model with
coefficients on parameters such as volatility, average daily volume, and
turnover that have been fitted on institutional trade data in a similar manner
as discussed by Almgren \cite{almgren2005direct}.

% Account for varying book sizes to see how that effects market impact
% and profitability



%%%%%%%%%%%%%%%%%%%%%%%%%%%%%%%%%%%%%%%%%%%%%%%%%%%%%%%%%%%%%%%%%%%%
\newpage
\section{Conclusion}

% TODO: Complete conclusion


%%%%%%%%%%%%%%%%%%%%%%%%%%%%%%%%%%%%%%%%%%%%%%%%%%%%%%%%%%%%%%%%%%%%
\newpage
%TODO: Ensure formatting matches submission format
\printbibliography

\end{document}
