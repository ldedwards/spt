\documentclass[british,english]{amsart}
\usepackage{lmodern}
\usepackage{biblatex}
\addbibresource{spt.bib}
\renewcommand{\sfdefault}{lmss}
\renewcommand{\ttdefault}{lmtt}
\usepackage[T1]{fontenc}
\usepackage[latin9]{inputenc}
\usepackage{verbatim}
\usepackage{amstext}
\usepackage{amsthm}
\usepackage{amssymb}

\makeatletter
%%%%%%%%%%%%%%%%%%%%%%%%%%%%%% Textclass specific LaTeX commands.
\numberwithin{equation}{section}
\numberwithin{figure}{section}
\theoremstyle{plain}
\newtheorem{thm}{\protect\theoremname}[section]
  \theoremstyle{definition}
  \newtheorem{defn}[thm]{\protect\definitionname}
  \theoremstyle{plain}
  \newtheorem{assumption}[thm]{\protect\assumptionname}
  \theoremstyle{plain}
  \newtheorem{lem}[thm]{\protect\lemmaname}
  \theoremstyle{plain}
  \newtheorem{prop}[thm]{\protect\propositionname}

%%%%%%%%%%%%%%%%%%%%%%%%%%%%%% User specified LaTeX commands.
\usepackage{color}
\usepackage[colorlinks=true,linkcolor={blue}]{hyperref}

\usepackage{amsfonts}
%\renewenvironment{lyxgreyedout}{\color{red}\bgroup}{\egroup}

\makeatother

\usepackage{babel}
  \addto\captionsbritish{\renewcommand{\assumptionname}{Assumption}}
  \addto\captionsbritish{\renewcommand{\definitionname}{Definition}}
  \addto\captionsbritish{\renewcommand{\lemmaname}{Lemma}}
  \addto\captionsbritish{\renewcommand{\propositionname}{Proposition}}
  \addto\captionsbritish{\renewcommand{\theoremname}{Theorem}}
  \addto\captionsenglish{\renewcommand{\assumptionname}{Assumption}}
  \addto\captionsenglish{\renewcommand{\definitionname}{Definition}}
  \addto\captionsenglish{\renewcommand{\lemmaname}{Lemma}}
  \addto\captionsenglish{\renewcommand{\propositionname}{Proposition}}
  \addto\captionsenglish{\renewcommand{\theoremname}{Theorem}}
  \providecommand{\assumptionname}{Assumption}
  \providecommand{\definitionname}{Definition}
  \providecommand{\lemmaname}{Lemma}
  \providecommand{\propositionname}{Proposition}
\providecommand{\theoremname}{Theorem}

% ------ DEFINITIONS -----
% Define the set of progressively measurable processes L
\renewcommand{\L}{\mathcal{L}}

\begin{document}
\title{Examination of practical applications of Functionally Generated 
Portfolios in Stochastic Portfolio Theory}
\author{Lawrence Edwards}
\maketitle

\tableofcontents{}

\section{Introduction}

Stochastic Portfolio Theory was first introduced by Fernholz and Shay in 
\cite{fernholz1982} and further developed by Fernolz in \cite{fernholz2002}.

\section{Notation}

From talk by Ruf \cite{karatzas2017}.

A standard probability space $(\Omega,\mathcal{F},P)$ equipped with a 
right-continuous filtration $\mathfrak{F}$.


$\L(X)$: denotes the set of all progressively measurable processes, 
integrable with respect to some semimartingale $X(\cdot)$.

$d\in\mathbb{N}:$ number of assets at time zero.

\textit{Nonnegative} continuous $P$-semimartingales, representing the relative market 
weights of each asset:

\begin{equation}
	\mu(\cdot) = \left( \mu_{1}(\cdot),\ldots,\mu_{d} \right)^{T}
\end{equation}

taking values in the unit simplex, i.e. all market weights are positive and sum 
up to $1$.

\begin{equation}
\triangle^{d}=\left\{(x_{1}(\cdot),\ldots,x_{d})\in[0,1]^{d}:\sum_{i=1}^{d}x_{i}=1\right\}.
\end{equation}

Note that we immediately place ourselves into the price process expressed as the 
total market capitalisation.

\subsection{Trading Strategies}

\newcommand{\V}{\mathcal{V}}

Given: $\V\in\L(\mu)$.

Define:

\begin{equation}
	V^{\V}(\cdot)=\sum_{i=1}^{d}\V_{i}(\cdot)\mu_{i}(\cdot)
\end{equation}

\printbibliography
\end{document}
